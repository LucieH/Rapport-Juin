\section{Présentation de l'application et des exercices}
L'application développée dans le cadre de ce TFE s'appelle \textit{Manabu}. Il s'agit d'un jeu didactique ayant pour but d'aider les enfants dans leur apprentissage de la lecture. J'ai choisi de lui donner le nom de \textit{Manabu}, qui signifie \textit{apprendre} en Japonais. Je trouve que ce nom sonne bien, est facile à retenir, et correspond au thème de l'application.\\

L'application \textit{Manabu} est composée de quatre exercices : \textit{Imagerie}, \textit{Lecture flash}, \textit{Anagrammes}, \textit{Ecouter le son}. Ceux-ci sont basés sur les conseils d'une institutrice primaire et d'une logopède. Ils s'inspirent de plusieurs méthodes d'apprentissage, principalement la méthode syllabique et  la méthode naturelle. Celles-ci ont été expliquées dans le point précédent, \textit{Recherches documentaires}.\\

Le premier exercice, \textit{Imagerie}, est inspiré de la pédagogie Freinet\footnote{Explications concernant cette pédagogie au point \textit{\ref{Freinet}}}. Je me suis inspirée d'un outil utilisé dans les classes de primaire : les fichiers freinet. Ces fichiers sont composés d'exercices d'évaluation dans différentes matières, telles que le français et les mathématiques. Dans le cas présent, j'ai pris pour influence celles consacrées à l'évaluation de la lecture, pour la première primaire.\\

Le principe de l'exercice est simple : une image représentative est associée à la lecture d'un mot ou d'une phrase. Une fois le mot lu, l'enfant retourne la fiche. Au dos de celle-ci se trouve la même image, où une image proche exprimant la même idée, ainsi que trois choix. L'enfant doit choisir parmi ces trois choix lequel correspond à ce qu'il a lu précédemment, en s'aidant de l'image. Il existe également une version plus compliquée, basée sur le principe de discrimination. Dans ce cas, deux mots/phrases sont présentées à l'enfant, accompagnés de deux images. Au dos se trouvera alors une troisième image, représentant le choix différent des deux lus précédemment. Il s'agit de celui à choisir.\\

Le deuxième exercice, \textit{Lecture flash}, a été décidé après ma rencontre avec Mme Henrion, la logopède. Le principe de ce type de lecture est assez simple. Comme son nom l'indique, il s'agit de lire un mot rapidement, et de le mémorisé. La durée de lecture du peut varie en fonction des facilités, ou difficultés, de l'enfant. Une fois le mot lu et mémorisé, il est demandé d'écrire celui-ci (ou taper dans le cas présent) à l'endroit prévu à cet effet. Pour un enfant plus expérimenté, le principe peut être appliqué à des phrases plus ou moins longues.\\

L'avantage de cet exercice est qu'il travaille non seulement les capacités de lecture de l'enfant, mais également celles de restitution ainsi que l'orthographe. Ceci permet notamment une meilleure mémorisation des mots et le travail de la rapidité de lecture. Le but du jeu étant évidemment de réussir à lire un maximum, en un minimum de temps. \\

Le troisième exercice est \textit{Anagrammes}. J'ai également décidé de mettre en place cet exercice sur base des conseils de la logopède. Celui-ci fonctionne de la manière suivante : les lettres d'un mot sont mélangées de manière aléatoire. L'enfant, lors de l'affichage des lettres, entend le mot qu'il doit reconstituer. Il lui faut remettre les lettres dans l'ordre afin de compléter l'exercice. Bien entendu, l'enfant peut ré-écouter le mot autant de fois qu'il lui est nécessaire.\\

Le challenge de cet exercice réside en premier dans la reconnaissance des lettres, et ensuite dans la connaissance de l'orthographe des sons et des mots. En effet, certains sons sont composés de plusieurs lettres, tels que \textit{ou, au, ai,} etc. Pour aider l'enfant dans la reconnaissance des lettres, les voyelles et les consonnes sont de couleurs différentes, ce qui permet déjà une première discrimination.\\

Enfin, le quatrième et dernier exercice, \textit{Ecoute le son}, est une application de la méthode syllabique et de la méthode alphabétique. Pour rappel, ces méthodes d'apprentissage se basent sur la reconnaissance des sons et des lettres pour enseigner la lecture.\\

L'exercice fonctionne de la manière suivante : plusieurs mots sont affichés à l'écran, et le son d'une syllabe est prononcé. L'enfant doit retrouver parmi les propositions le mot dans lequel se trouve la syllabe. Pour le niveau le plus facile, la syllabe se trouve soit au début, soit à la fin du mot. Ensuite, la difficulté augmente, et il s'agit de trouver le son au milieu d'un mot de minimum 3 syllabes.



% Présenter les exercices et l'application de manière commerciale et attractive !!!