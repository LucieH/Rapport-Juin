\section{Analyse du problème}
Avant de me lancer dans la partie technique et la programmation de ce travail de fin d'études, il m'a fallu effectuer quelques recherches. En effet, ne s'improvise pas instituteur qui veut. Bien que nous ayons tous dans notre vie appris à lire, il m'était impossible d'imaginer concevoir sans aucune aide une application supposée aider les enfants dans leur apprentissage de la lecture. J'ai donc choisi de me documenter auprès de professionnelles travaillant avec des enfants. Afin de valider la pertinence de la problématique, je me suis également renseignée sur les applications existantes, tout en recoupant leur contenu avec les informations obtenues précédemment.

Ce point est consacré aux recherches que j'ai effectuées avant de commencer l'application en elle-même. Je présenterai tout d'abord les informations obtenues auprès de l'institutrice et de la logopède, ainsi que la conclusion de ces deux entrevues. J'analyserai ensuite les produits existants sur le marché sur la base des informations obtenues. Je terminerai avec le cahier des charges établi pour l'application suite à ces recherches.

\subsection{Rencontre avec des professionnelles de l'apprentissage de la lecture}
J'ai choisi de commencer mon travail par une étape de documentation auprès de professionnelles travaillant avec des enfants. Pour ce faire, j'ai eu recours à l'aide et l'expérience de Mme Odile Paveau, institutrice primaire à l'école des Bruyères de Louvain-La-Neuve, ainsi que de Mme Laurence Henrion, logopède exerçant à Louvain-La-Neuve également. Toutes deux m'ont dit être intéressées par le sujet de mon TFE. Elle ont accepté de me fournir des explications, des conseils, et des pistes pour le bon développement de mon application Android. Ci-dessous se trouvent les synthèses des entrevues; le compte-rendu complet des rencontres figure quant à lui dans l'annexe \ref{annexeInterview}.

\subsubsection{Rencontre avec une institutrice : Odile Paveau\label{Freinet}}
J'ai rencontré Odile Paveau le 16 décembre 2014 dans la classe de primaire où elle enseigne, à l'école des Bruyères de Louvain-La-Neuve. Lors de cette entrevue, trois grands points ont été abordés : les méthodes d'apprentissage les plus utilisées pour la lecture, les spécificités de l'apprentissage à l'école des Bruyères, et la présentation d'outils utilisés ainsi que des exemples d'exercices.\\

Il existe bon nombre de méthodes d'apprentissage, qui peuvent être très variées. Parmi celles-ci, Odile a fait mention de trois méthodes en particulier, et qui concernent l'apprentissage de la lecture :
\begin{itemize}
\item la méthode globale,
\item la méthode syllabique,
\item la méthode naturelle.\\
\end{itemize} 

La méthode globale consiste à apprendre à lire à partir du mot en entier. Le but de cette méthode est \textit{"de faire acquérir à l'élève une stratégie de déchiffrage des mots, voire des phrases, en tant qu'image visuelle indivisible"}\footnote{Citation tirée de l'article \textit{Méthode globale}, http://fr.wikipedia.org/wiki/Méthode\_globale}. En pratique, cela signifie que l'enfant, la première fois qu'il rencontre le mot, est invité à le deviner. Celui-ci mémorisera alors le mot en le rencontrant plusieurs fois dans des contextes différents (chansons, petites histoires, poèmes, ...). Le mot est dès lors associé à une idée, ce qui permet de qualifier cette méthode d'\textit{idéovisuelle}.\\

La méthode syllabique, par opposition à la méthode globale, part des sons que forment les lettres et les syllabes afin de construire le mot. Celle-ci relie la phonétique des lettres avec l'alphabet afin de construire tout d'abord les syllabes, puis d'assembler ces dernières pour créer les mots. Cette méthode se base sur le décryptage progressif des phrases lues.\\

La méthode naturelle s'inspire quant à elle de la pédagogie Freinet. Cette dernière, mise au point par Célestin Freinet au siècle dernier, est fondée sur l'expression de la créativité des enfants. Il s'agit de donner à l'enfant un projet, qui lui sera utile dans son apprentissage, et qui prend en compte ses centres d'intérêts et le potentiel créatif et associatif de celui-ci. Du point de vue de l'apprentissage de la lecture, cette méthode implique de partir du sens des mots, afin de donner un sens à ce qui est appris. La collaboration de tout le groupe est nécessaire, et l'essai-erreur est appliqué. La pédagogie Freinet a d'abord été associée à la méthode globale. Cependant, son procédé différant dans la façon dont les mots sont appris, l'apprentissage de la lecture par cette pédagogie se nomme désormais méthode naturelle.\\

A l'école des Bruyères, les instituteurs ne se concentrent pas sur une méthode en particulier pour enseigner la lecture aux enfants. Ce sont les méthodes syllabique, pour son apprentissage fluide, et naturelle qui sont principalement utilisées, comparativement à "l'ancienne technique" des écoles primaire, dans laquelle la méthode globale primait. Néanmoins, la méthode globale n'est pas totalement remisée au placard, car elle est encore d'application dans certains exercices. Elle reste cependant peu utilisée, étant considérée comme lourde et peu optimale, à contrario des deux autres qui peuvent être plus ludiques. En effet, il est très important de faire sens pour l'enfant et de susciter son intérêt afin de faciliter l'apprentissage.\\

Pour apprendre à lire et mettre en pratique ces méthodes, les enfants utilisent divers outils. Tout d'abord, notons que la lecture s'apprend en premier lieu à l'aide de caractères imprimés en majuscules, puis en minuscules, et enfin avec la police de type écriture manuelle, appelée aussi \textit{cursive}. Ceci permet aux enfants de se familiariser petit à petit avec les lettres, celles-ci présentant des motifs plus complexes au fil des étapes. Ensuite, l'enfant travaille beaucoup sur ses centres d'intérêts. Il faut le faire travailler sur des sujets qui le concernent ou l'amusent tels que son prénom, son âge, ce qu'il aime, des phrases rigolotes, etc.\\

Durant l'apprentissage, le rapport au son est très important pour l'enfant. Voici quelques exercices et outils qui sont proposés à l'école :
\begin{itemize}
\item Avec des lettres d'imprimerie disposées dans n'importe quel sens\footnote{Le fait de mettre les lettres droites, à l'envers, ou de côté permet d'entraîner l'enfant à différencier les caractères.}, l'institutrice donne un son, et il faut retrouver la lettre qui produit ce son.
\item Classer les prénoms des enfants de la classe en les rassemblant en fonction des sons par lesquels ils commencent. Ceci permet notamment de découvrir de nouveau phonèmes, comme dans le prénom \textit{Hugo}, où le son entendu est \textit{U} et le phonème associé est \textit{Hu}.
\item Dans un petit texte, qui peut être écrit sur la base d'une idée de l'enfant (par exemple "\textit{Marie aime la danse et les poupées}"), retrouver les sons déjà connus et les souligner en couleur. La couleur aide ici et à mémoriser, et à différencier.
\item Dans un texte, trouver le son qui revient le plus souvent. Cet exercice est une variante du précédent.
\item A partir d'un son, trouver des mots qui commencent par celui-ci, par exemple sous forme d'images pour ensuite voir comment s'écrit le mot. Au niveau supérieur, le son peut se trouver au milieu ou en fin de mot.
\item Comme outil, créer un dictionnaire \textit{référent}. Celui-ci reprend, pour chaque son (également les composés type \textit{au, ou, en, ...}) un dessin représentatif du son (par exemple une chouette pour le son \textit{ch}) ainsi qu'une liste des mots commençant par ce son.
\item Retrouver un mot connu dans un texte (ici, la méthode globale est utilisée).
\item Etc.\\
\end{itemize}

Lors des évaluations de l'apprentissage de la classe, la méthode naturelle est la plus souvent utilisée, car c'est celle qui fait le plus appel à l'imagination de l'enfant. Comme outil, les instituteurs utilisent notamment les fichiers Freinet. Créés sur base de la pédagogie Freinet, ces fichiers sont composés d'images associées à des mots de différents niveaux. L'idée est d'associer un (ou plusieurs) mot(s) lu(s) à une (ou plusieurs) image(s), et de vérifier les capacités de lecture de l'enfant en lui proposant soit le même mot à retrouver dans une liste sur base d'une image semblable à la précédente, soit un mot différent sur base d'une image différente des deux précédentes, soit encore l'association d'un nouveau mot en complément avec celui lu précédemment, sous forme d'une phrase. Des exemples de fichiers se trouvent à l'annexe \ref{annexeFreinet}. Ces fichiers existent également pour d'autres matières, comme les mathématiques.\\

Enfin, à l'école des Bruyères, l'enfant à toujours accès à une boîte à outils pour s'aider en cas de difficultés. Celle-ci est composée du dictionnaire référent, mais aussi de panneaux et d'affiches se trouvant un peu partout dans la classe. Ceux-ci rappellent les couleurs identifiants les sons, les images associées, les dessins des enfants représentant des mots, des symboles associés aux lettres, etc.

\subsubsection{Rencontre avec une logopède : Laurence Henrion}
J'ai rencontré Laurence Henrion le 19 janvier 2015. Elle a accepté de me recevoir dans le cabinet où elle exerce son activité de logopède. Durant notre entrevue, elle a mentionné différents points que je considère comme importants, et que je détaillerai ci-dessous : l'importance de l'âge et les connaissances de bases pour la lecture, les exercices qu'elle utilise en tant que logopède et les applications spécialisées, et enfin les détails pratiques pouvant m'aider pour la programmation de mon application.\\

Comme expliqué précédemment, il existe plusieurs méthodes pour apprendre à l'enfant à lire. L'idéal, cependant, est de commencer à lui inculquer les bases de la lecture vers 3 ou 4 ans. En effet, ceci permet d'optimiser ses compétences par la suite. Ce principe vient notamment des méthodes pédagogiques proposées par Maria Montessori\footnote{Pour plus d'informations : \textit{Pédagogie Montessori}, \url{http://fr.wikipedia.org/wiki/Pédagogie_Montessori}}. Cette pédagogie met l'accent sur l'importance des périodes sensibles de l'enfant\footnote{Une période sensible est un âge où l'enfant est plus prompt à développer certains aspect de son évolution naturellement.}. Celle concernant le langage se déroule environ de 2 mois à 6 ans. La pédagogie Montessori part du principe qu'il existe une conscience phonologique fort présente chez l'enfant à ce moment. Celui-ci apprend beaucoup à l'aide de rimes, de comptines, etc. C'est-à-dire à l'aide de sons. Ceci rejoint ce qui m'avait déjà été dit par Odile à propos de la méthode syllabique : il est important de se baser sur le son que fait la lettre seule, ou le groupe de lettres, et non le nom qui lui est donné, car ce dernier est source de confusions.\\

Néanmoins, être dans la bonne période ne suffit pas pour maîtriser l'art de la lecture. En effet, pour apprendre à lire, l'enfant doit déjà maîtriser un certain vocabulaire à l'oral. Dès lors, si le vocabulaire de base n'est pas acquis, l'enfant ne sera pas capable de comprendre ce qu'il lit. Les premiers mots et textes seront dont composés de vocabulaire basique, pas de mots compliqués et peu fréquents, tels que \textit{narval}, \textit{okapi}, etc.\\

En tant que logopède, Laurence Henrion travaille avec différents exercices basés sur le son. Certains exercices rejoignent ce qui avait été dit aux Bruyères, notamment :
\begin{itemize}
\item La sélection d'un mot contenant un son entendu précédemment parmi plusieurs choix.
\item Choisir une lettre qui produit le son prononcé. Ici, Laurence Henrion m'a spécifié que la lettre importait peu tant que le son était correct. Par exemple, pour le son \textit{sss}, l'enfant peut choisir aussi bien le \textit{s} que le \textit{c} ou le \textit{ç.}
\item Relier un mot ou un son à une image.
\item Associer les sons et les lettres à des couleurs qui leur seront spécifiques.\\
\end{itemize}

Un autre exercice qu'elle affectionne particulièrement est la lecture flash.
Cet exercice consiste à mémoriser un mot (ou une phrase) affiché plus ou moins rapidement à l'écran, pour ensuite le réécrire correctement. Dans le cas présent, l'orthographe est travaillée en plus de la lecture. Ce sont deux matières très liées. Un jeu de lecture flash doit idéalement :
\begin{itemize}
\item avoir un temps réglable d'affichage selon le niveau de l'enfant,
\item insister sur l'importance de la mémorisation,
\item faire attention au sens de la lecture.
\end{itemize}
En plus de cela, l'exercice peut être amélioré grâce à :
\begin{itemize}
\item l'affichage de mots personnalisés en fonction des centres d'intérêt de l'enfant,
\item la lecture effectuée avec un mot, et la réécriture sous la forme d'une image à choisir.\\
\end{itemize}

Laurence Henrion utilise beaucoup son iPad pour travailler avec les enfants. En effet, c'est plus fun pour l'enfant de jouer avec une tablette.
La majorité des applications présentes sur l'iPad ont été créés par un spécialiste orthophoniste (appelation d'un logopède en France) : Emmanuel Crombez. Celui-ci réalise des applications pour iPad, iPhone et Mac sous le nom \textit{ABC Applications}. Celles-ci sont destinées à aider les enfants dans leur apprentissage général et dans leurs difficultés : lecture, écriture, mathématiques, etc. Parmi ces jeux, on peut trouver\footnote{Liste complète sur le site d'Emmanuel Crombez : \url{http://abc-applications.com/ipad.html}.} :
\begin{itemize}
\item \textit{Anagrammes}. L'enfant doit retrouver des mots dont les lettres ont été mélangées. Au besoin, il peut s'aider en écoutant le mot qu'il doit reconstituer.
\item \textit{Nuages de mots}. Cet exercice propose à l'enfant de retrouver le genre et le nombre des mots affichés en "nuages". Quel mot est masculin, ou féminin ? Est-ce singulier ou pluriel ? Le jeu entraîne à la lecture mais aussi au classement. Ce jeu existe aussi sous la forme \textit{Nuages de lettres}, adapté aux enfants de maternelle pour apprendre à reconnaître les lettres.
\item \textit{Mémo des mots}. Ce jeu se présente sous la forme du bien connu \textit{Memory}. La différence étant qu'ici des mots sont utilisés à la place des images.
\textit{Etc.}\\
\end{itemize}

Pour terminer, j'ai reçu quelques conseils concernant ce qui était préférable d'éviter pour mon application, et ce qui pouvait m'être utile. Tout d'abord, il est déconseillé d'inclure trop d'éléments graphiques. Il vaut mieux opter pour un design simple et épuré. Trop d'objets sur l'écran, c'est prendre le risque que l'enfant soit distrait et ne se concentre pas sur ce qui est requis, donc perdre en efficacité. Ensuite, il existe une police open source conçue pour faciliter la lecture chez les personnes dyslexiques en accentuant certaines parties des lettres. Celle-ci se nomme \textit{OpenDyslexic}. Dans le cas où celle-ci n'aurait pas été disponible, un police de type sans serif telle que \textit{Verdana} est idéale. En dernier lieu, Laurence m'a conseillé d'utiliser le VOB (Vocabulaire Orthographique de Base) pour les exercices que je mettrais en place dans l'application. Il s'agit de la liste officielle en Belgique des mots qui doivent être connus et maîtrisés par les enfants à la fin de chaque cycle de primaire.

\subsection{Conclusion des entrevues}
Les interviews avec Odile Paveau et Laurence Henrion m'ont permis de comprendre les méthodes utilisées dans l'apprentissage de la lecture chez l'enfant. Elles m'ont expliqué ce qui était le plus efficace, ainsi que les erreurs que je devais éviter. Voici ce que j'en ai retiré.\\

Premièrement, je me suis renseignée un peu plus sur les différentes méthodes d'apprentissage afin de choisir ce que j'allais développer comme exercices. Je n'ai pas étudié le sujet en profondeur, l'essentiel pour moi étant de bien comprendre le fonctionnement des méthodes et ce que je pouvais en retirer. Voici en quelques mots une synthèse des différentes méthodes et pédagogies qui ont été abordées :
\begin{itemize}
\item la méthode globale : apprendre à lire à partir du mot en entier. C'est une méthode idéovisuelle qui veut que l'enfant mémorise les mots rencontrés sous forme d'image et d'idée, et puisse les reconnaître plus tard;
\item la méthode syllabique : opposée à la méthode globale, elle part du son des syllabes et des lettres pour former les mots. Il s'agit ici de déchiffrer;
\item la méthode naturelle inspirée de la pédagogie Freinet : elle part du sens des mots et des intérêts de l'enfant. L'idée est de stimuler l'enfant, sa créativité, ce qu'il aime, afin de mémoriser le mot;
\item la pédagogie Montessori : met l'accent sur les périodes sensibles de l'enfant et l'apprentissage spécifique lié à celles-ci. Pour la période liée au langage et à la lecture, il existe une conscience phonologique fort présente chez l'enfant : il apprend beaucoup grâce aux comptines, chansons, rimes, etc.
\end{itemize}
Tant Odile que Laurence ont insisté sur l'intérêt que doit porter l'enfant aux jeux et exercices, ainsi que sur l'efficacité de travailler avec les sons des lettres et les syllabes. Il est donc évident pour moi que les méthodes idéales pour une application sont la méthode syllabique et la méthode naturelle. Concernant la pédagogie Montessori, je n'ai pas de prise sur l'âge des utilisateurs.\\

À côté des méthodes d'apprentissage, différents éléments ressortent des entrevues. Tout d'abord, le point le plus important est de susciter l'intérêt de l'enfant pour l'apprentissage de la lecture. Éveiller sa curiosité en choisissant des sujets qui l'intéressent ou en utilisant des exercices ludiques lui permet d'apprendre plus efficacement.\\

D'autre part, certains exercices m'ont paru plus importants, ou plus intéressants comparés à d'autres. Le premier est celui où il faut retrouver les sons de lettres ou de syllabes dans les mots. Autant Odile que Laurence m'en ont parlé comme d'une "base" dont l'efficacité a été prouvée.
Certains des autres exercices mentionnés me paraissent également adaptables pour mon application :
\begin{itemize}
\item les fichiers Freinet pour la lecture,
\item retrouver des mots dans un texte,
\item la lecture flash, sur laquelle Laurence Henrion a particulièrement insisté,
\item les anagrammes,
\item les nuages de mots.\\
\end{itemize}

Enfin, le dernier point, très important à mes yeux, est le conseil d'utilisation de la police spécialisée \textit{OpenDyslexic} pour la lisibilité du texte de l'application, ainsi que l'emploi du VOB pour le vocabulaire proposé.

\subsection{Étude des produits existants et critique sur la base des entrevues}
À la suite des entrevues avec Odile Paveau et Laurence Henrion, j'ai étudié les produits existants sur le marché dans le domaine de l'apprentissage de lecture. J'ai exploré diverses plateformes afin de comparer ce qui existe déjà, et réfléchi à ce que j'avais la possibilité d'y apporter, tout en recoupant le contenu des applications et mes nouvelles connaissances.\\

J'ai concentré mes recherches sur les \textit{stores} respectifs d'\textit{Apple} et d'\textit{Android}. Je ne me suis pas attardée sur le \textit{Windows Store}, celui-ci étant selon moi encore trop peu fourni. J'ai constaté qu'il existe déjà des applications aidant les enfants dans leur apprentissage de la lecture, aussi bien pour les appareils Android que pour les iPads. Cependant, la plupart des applications disponibles le sont pour des langues autres que le Français. Il n'existe pas un grand choix d'applications en Français qui soient complètes tout en étant vraiment amusantes pour les enfants, surtout sous Android. Comparativement, il va sans dire que l'offre de l'Apple Store est plus variée, notamment grâce aux applications d'Emmanuel Crombez, non disponibles pour Android.\\

J'ai parcouru une série d'applications, notamment celles que Mme Van den Schrieck a installées pour ses enfants sur son iPad. La plupart ont un aspect très ludique et amusant. Malheureusement, avec ce que j'ai appris concernant les méthodes d'apprentissages de la lecture, certaines me semblent trop peu structurées et sans méthodologie concrètement définie.\\

Une application disponible sur les deux plateformes qui m'avait plu et que je considère comme la mieux conçue s'appelle \textit{Lire avec Sami et Julie}. En effet, celle-ci favorise la méthode syllabique, ainsi que naturelle. Elle travaille sur le son, avec des images, et crée ainsi des associations chez l'enfant. Cependant, \textit{Lire avec Sami et Julie} est une application au design et à l'ambiance sonore trop chargés. Cela n'aide pas l'enfant à se concentrer, comme me l'a expliqué Laurence Henrion.\\

Bien évidemment, les applications d'Emmanuel Crombez sont de très bonne facture. Le seul reproche que j'ai à leur propos est que chaque exercice représente une application séparée. Mon but est dès lors de réaliser une application qui soit plus complète que celles que j'ai pu rencontrer. J'entends par là qu'elle offre une plus grande variété d'exercices pour l'enfant, se basant sur les méthodes expliquées lors des entrevues, et appliquant les conseils reçus.

\subsection{Cahier des charges}
Sur la base des entrevues ainsi que de l'étude des produits existants sur le marché, j'ai établi un cahier des charges. Celui-ci reprend les différents points que j'aimerais mettre en place dans mon application, les spécificités de celles-ci, etc.\\

Tout d'abord, l'application présentera un design simple et épuré au maximum. Je ne veux pas pour autant que ce soit laid ou ennuyeux. La couleur de fond de l'application sera le bleu. En effet, le bleu plaît à une majorité de la population et inspire le calme. Il s'agit d'une couleur établissant un bon compromis entre esthétique et ambiance pour se concentrer.\\

Toujours afin d'appliquer les conseils reçus, je ferai d'\textit{OpenDyslexic} la police principale de l'application. Dans le cas où je n'y parviendrais pas, j'opterai pour du sans serif.\\

L'application sera composée de quatre exercices. Ceux-ci seront inspiré de la méthode syllabique ainsi que de la méthode naturelle, qui semblent être celles préconisées dans le domaine de l'apprentissage de la lecture. Ces exercices seront basés sur ce qui est utilisé par les instituteurs des Bruyères, ainsi que par la logopède. J'ai décidé de mettre en place :
\begin{itemize}
\item un exercice inspiré des fichiers Freinet utilisés pour les évaluations de la lecture. J'aime beaucoup l'association image et mot;
\item un exercice de lecture flash. Laurence Henrion a particulièrement insisté sur cet exercice lors de notre entrevue;
\item un exercice d'anagrammes;
\item un exercice de repérage du mot qui contient un son en particulier.
\end{itemize}
Tous ces exercices utiliseront comme base de vocabulaire le VOB.\\

Concernant les exercices, la première étape est de les implémenter avec chacun un niveau basique. Dans l'idéal, si le développement se passe bien, j'ajouterai des niveaux de difficulté. Dans ce cas, chaque exercice aurait 3 niveaux : facile, moyen et difficile.\\

Enfin, j'aimerais que l'application soit ludique, afin que les enfants aient envie de l'utiliser par eux-même. Ceci passe d'abord par les exercices qui doivent leur paraître "fun". Pour ce faire, j'utiliserai des bruitages, par exemple lors de la validation d'une réponse. J'aimerais également intégrer une mascotte à l'application, si possible.