\section{Recherches documentaires}
Avant de me lancer dans la partie technique et programmation de ce travail de fin d'études, il m'a fallu effectuer quelques recherches. J'ai commencé par valider la pertinence du sujet avec une étude comparative des solutions existantes sur le marché. J'ai exploré diverses plateformes afin de comparer ce qu'il existe déjà, ainsi que ce que j'avais la possibilité d'y apporter. \\

Par ailleurs, ne s'improvise pas instituteur qui veut. Bien que nous ayons tous dans notre vie appris à lire, il m'était impossible d'imaginer concevoir sans aucune aide une application supposée aider les enfants dans leur apprentissage de la lecture. J'ai donc choisi de commencer mon travail par une documentation auprès de professionnels travaillant avec des enfants. Pour ce faire, j'ai eu recours à l'aide et l'expérience de Mme Odile Paveau, institutrice primaire à l'école des Bruyères de Louvain-La-Neuve, ainsi que Mme Laurence Henrion, logopède exerçant à Louvain-La-Neuve également. Toutes deux m'ont fourni des explications, des conseils, et des pistes pour le bon développement de mon application android.\\

Ce point est consacré aux recherches que j'ai effectuées avant de commencer l'application en elle-même. Ci-dessous se trouvent les détails de l'étude comparative des produits, ainsi que les comptes-rendus des interviews et les informations obtenues lors de ces dernières.

\subsection{Etude comparative du marché}
Va falloir retrouver ton blabla lucie... Reprendre ce que j'avais noté en décembre. Note de la défense de décembre :\\

Il existe déjà des applications aidant les enfants dans leur apprentissage de la lecture, aussi bien pour les appareils Android que pour les iPads. Cependant, la plupart des applications disponibles sont pour des langues autres que le Français. Il n'existe pas un grand choix d'application en Français qui soient vraiment complètes tout en étant vraiment amusantes pour les enfants.\\

L'application que j'ai trouvée la mieux conçue pour le moment est celle Lire avec Sami et Julie. En effet, celle-ci favorise la méthode syllabique, une de celles appliquées à l'école primaire par les enseignants. J'aimerais réaliser une application qui soit plus complète que celle-ci. J'entends par là qu'elle offre une plus grande variété d'exercices pour l'enfant, se basant à la fois sur la méthode syllabique et la méthode globale (celles-ci sont expliquées par la suite).

\subsection{Rencontre avec Odile Paveau\label{Freinet}}
J'ai rencontré Odile Paveau dans sa classe de primaire le 16 décembre 2014 dans la classe de primaire où elle enseigne, à l'école des Bruyères de Louvain-La-Neuve. \\

Elle m'a expliqué ce qu'elle, en temps qu'institutrice primaire, utilise comme méthodes pour enseigner la lecture. Elle m'a de même expliqué l'"ordre des choses" qu'elle met en place pour préparer les enfant à la lecture, et l'intégrer petit à petit dans leur cerveau.


\subsection{Rencontre avec Laurence Henrion}
J'ai rencontré Laurence Henrion le 19 janvier 2015. Elle a accepté de me recevoir dans le cabinet où elle exerce son activité de logopède pour répondre à mes questions.\\

Elle a commencé par m'expliquer qu'il existe plusieurs méthodes pour apprendre à l'enfant à lire. L'idéal, cependant, est de commencer à lui inculquer les bases de la lecture vers 3 ou 4 ans. En effet, ceci permet d'optimiser ses compétences par la suite. La méthode proposant de commencer l'apprentissage si tôt s'appelle \textit{Montessori}. Celle-ci part du principe qu'il existe une conscience phonologique fort présente chez l'enfant. Celui-ci apprend beaucoup à l'aide de rimes, de comptines, etc. C'est-à-dire à l'aide de sons. Ceci rejoint ce qui m'avait déjà été dit par Odile : il est important de se baser sur le son que fait la lettre seule, ou le groupe de lettres, et non le nom qui lui est donné, car ce dernier apporte des confusions.\\

De plus, elle a insisté sur le point suivant : pour apprendre à lire, l'enfant doit déjà maîtriser un certain vocabulaire à l'oral. Dès lors, si le vocabulaire de base n'est pas acquis, l'enfant ne sera pas capable de comprendre ce qu'il lit. Les premiers mots et textes seront dont composés de vocabulaire basique, pas de mots compliqués et peu fréquents, tels que \textit{narval}, \textit{okapi}, etc.