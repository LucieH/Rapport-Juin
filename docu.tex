\section{Recherche documentaire}
Avant de me lancer dans la partie technique et programmation de ce travail de fin d'études, il m'a fallu effectuer quelques recherches. En effet, ne s'improvise pas instituteur qui veut. Bien que nous ayons tous dans notre vie appris à lire, il m'était impossible d'imaginer concevoir sans aucune aide une application supposée aider les enfants dans leur apprentissage de la lecture. J'ai donc choisi de commencer mon travail par une documentation auprès de professionnels travaillant avec des enfants. Pour ce faire, j'ai eu recours à l'aide et l'expérience de Mme Odile Paveau, institutrice primaire à l'école de Bruyères de Louvain-La-Neuve, ainsi que Mme Laurence Henrion, logopède exerçant à Louvain-La-Neuve également. Toutes deux m'ont fourni des explications, des conseils, et des pistes pour le bon développement de mon application android. Ce point est consacré au compte-rendu des interviews et reprend les informations qu'elle m'ont fournies.

\subsection{Rencontre avec Odile Paveau}
\subsection{Rencontre avec Laurence Henrion}