\section{Conclusion}
A la fin de ce rapport, je peux affirmer que je suis parvenue à réaliser une application Android permettant d'aider les enfants dans leur apprentissage de la lecture en dehors de l'école. Celle-ci est destinée aux enfants débutant dans la lecture, avec un niveau de première primaire ou même inférieur. \textit{Manabu} est une application conçue pour fonctionner principalement sur tablette, mais pouvant également être exécutée sur smartphone. Elle est à la fois conviviale, amusante et éducative. Les quatre exercices qui la composent sont inspirés de conseils venant de professionnelles de l'apprentissage de la lecture. Ils mettent en pratique deux méthodes d'apprentissage qui ont prouvé leur efficacité : la méthode syllabique et la méthode naturelle.\\

J'ai démontré tout au long de ce rapport le travail effectué. J'ai commencé par expliquer l'analyse que j'ai faite de la problématique, et les recherches en découlant. J'ai établi un cahier des charges et présenté l'application en conséquence. J'ai ensuite énuméré les outils que j'ai utilisés tout au long de ce travail de fin d'études. Par après, j'ai explicité le déroulement du développement de l'application, ainsi que les résultats obtenus après un test de l'application par des enfants. Enfin, j'ai établi la liste des difficultés que j'ai rencontrées durant ce projet, et j'ai énuméré quelques améliorations qu'il est possible d'apporter à l'application.\\

Enfin, je suis fière d'avoir réalisé \textit{Manabu}. C'est toujours un sentiment positif de parvenir à réaliser un projet de cette ampleur. J'ai rencontré des difficultés, tout ne s'est pas réalisé comme je l'avais prévu d'un claquement de doigt, mais ce sont les aléas de la programmation. J'ai également découvert le développement Android qui, bien qu'il utilise le langage Java, fonctionne par certains aspects différemment. En outre, le test de \textit{Manabu} par des enfants m'a confirmé que j'ai fait le bon choix de sujet de TFE. Ils ont apprécié les exercices et se sont amusés, ce qui pour moi est une grande réussite. 