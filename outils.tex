\section{Outils et technologies utilisés}
Ce point est consacré aux choix technologiques que j'ai effectués afin de mener à bien la réalisation de mon travail de fin d'études. Ceux-ci sont peu nombreux, compte tenu du fait qu'il est déjà possible d'effectuer beaucoup d'opérations avec les librairies de base d'Android. Par ailleurs, je n'avais pas besoin d'un grand nombre d'outils technologiques. Je commencerai par expliquer les spécifications du développement Android. Je détaillerai ensuite les outils utilisés tout au long du travail.\\

Le développement Android tel que je l'ai effectué est réalisé à l'aide de deux langages : XML et Java. Le XML est utilisé pour la description des interfaces graphiques de l'application. Ces fichiers sont appelés des layouts, ou encore parfois des vues. Le Java est quant à lui employé pour la programmation des actions à effectuer.\\

Lors du développement, la structure principale est l'application. Cette dernière inclut une ou plusieurs activités. Selon le site d'Android Developer, une activité se définit comme suit :\\
\textit{"An Activity is an application component that provides a screen with which users can interact in order to do something, such as dial the phone, take a photo, send an email, or view a map. Each activity is given a window in which to draw its user interface. The window typically fills the screen, but may be smaller than the screen and float on top of other windows."\footnote{Android Developer, Activities, \url{http://developer.android.com/guide/components/activities.html}}}\\
Une activité est donc la composante principale d'une application (sans au moins une activité, ça ne fonctionne pas).\\

Bien entendu, à côté des activités existent encore d'autres concepts. Une activité peut être associée à une ou plusieurs vues (définies par les layouts), ce qui fait que l'interface peut changer. Lors de l'élaboration des layouts, divers éléments peuvent êtres mis en place pour construire l'interface graphique (boutons, champs texte, images, etc.). Je ne me lancerai pas ici dans l'explication de tout ce que l'on peut trouver en Android. Ceci sera détaillé par la suite au besoin.\\

En ce qui concerne les outils, j'avais en premier lieu choisi de développer mon application avec l'IDE Eclipse Juno, comprenant le SDK Android afin de pouvoir programmer pour Android. Cependant, j'ai rencontré quelques problèmes techniques qui m'ont poussée à changer d'IDE. En effet, voulant utiliser une librairie externe avec une dépendance Maven, j'ai donc installé les outils permettant d'utiliser Maven avec Eclipse. Malheureusement, après l'installation des outils, Eclipse a strictement refusé de fonctionner plus de quelques minutes après chaque démarrage, m'empêchant de convertir mon projet existant pour l'utilisation de Maven, et bloquant au final. En raison de ce problème, j'ai décidé de passer d'Eclipse à Android Studio, considéré maintenant comme stable et efficace pour la programmation Android. En effet, Android Studio est à présent l'IDE officiel pour Android. Toutefois, celui-ci ne fonctionne plus à l'aide de dépendances Maven pour les librairies externes, mais grâce à Gradle. Après avoir importé mon projet Eclipse dans Android Studio sans encombre, j'ai constaté de Gradle qu'il est vraiment facile d'utilisation. J'ai donc choisi de continuer mon développement à l'aide de ce nouvel IDE, sans regrets.\\

De plus, toujours concernant la programmation, j'ai décidé de faire régulièrement des tests sur smartphone et tablette afin de vérifier la compatibilité de l'application \textit{Manabu} sur les différents appareils. J'avais à ma disposition : une tablette Samsung Galaxy 2 10.1, un smartphone Wiko Darkmoon, une tablette Samsung Galaxy Tab Pro et une tablette Nvidia Shield Tablet. La version du système d'exploitation Android est également différente entre les appareils : de la 4.0 Ice Cream Sandwich à la 5.1 Lollipop. Ceci m'assure de toucher un large public, mon application étant compatible à partir l'API 14, autrement dit 4.0, 4.0.1, 4.0.2 Ice Cream Sandwich. Par ailleurs, afin d'assurer la portabilité de mon travail entre les différents ordinateurs, j'ai créé des repositories sur GitHub, aussi bien pour le code que pour le rapport. Outre la portabilité, ceci me procure aussi un backup supplémentaire online.\\

Enfin, j'utilise divers outils, pas forcément technologiques, afin de compléter mon travail de fin d'études. Ceux-ci sont :
\begin{itemize}
\item Adobe Illustrator pour les graphismes, beaucoup plus adaptables lorsqu'ils sont réalisés en dessin vectoriel,
\item Audacity, un logiciel d'enregistrement et d'édition audio,
\item OpenDyslexic, une police de caractère adaptée pour faciliter la lecture,
\item le VOB (Vocabulaire Orthographique de Base) du premier degré, pour les mots que les enfants devront lire ou reconstituer lors des jeux.
\end{itemize} 



%RAJOUTER DESCRIPTION SPEC DEV ANDROID