\section{Outils et technologies utilisés}
Ce point se consacre aux choix technologiques que j'ai effectués afin de mener à bien la réalisation de mon travail de fin d'études. Ceux-ci sont peu nombreux, du fait qu'il est déjà possible d'effectuer beaucoup d'opérations avec les librairies de base d'Android. Par ailleurs, je n'avais pas besoin d'un grand nombre d'outils technologiques.\\

J'avais, en premier lieu, choisi de développer mon application avec l'IDE Eclipse Juno, comprenant le SDK Android afin de pouvoir programmer pour Android. Cependant, j'ai rencontré quelques problèmes techniques qui m'ont poussés à changer d'IDE. En effet, voulant utiliser une libraire externe avec une dépendance Maven, j'ai donc installé les outils permettant d'utiliser Maven avec Eclipse. Malheureusement, après l'installation des outils, Eclipse a strictement refusé de fonctionner plus de quelques minutes à chaque démarrage, m'empêchant de convertir mon projet existant pour l'utilisation de Maven, et bloquant au final. A cause de ce problème, j'ai décidé de passer d'Eclipse à Android Studio, considéré maintenant comme stable et efficace pour la programmation android. En effet, Android Studio est à présent l'IDE officiel pour Android. Toutefois, celui-ci ne fonctionne plus à l'aide de dépendances Maven pour les librairies externes, mais grâce à Gradle. Après avoir importé mon projet Eclipse dans Android Studio sans encombre, j'ai constaté de Gradle est vraiment facile d'utilisation. J'ai donc choisi de continuer mon développement à l'aide d'Android Studio, sans regrets.\\

De plus, toujours concernant la programmation, j'ai décidé de faire régulièrement des tests sur smartphone et tablette me permettant de vérifier la compatibilité de l'application \textit{Manabu} entre les différents appareils : une tablette Samsung Galaxy 2 10.1, un smartphone Wiko Darkmoon, et une tablette Nvidia Shield Tablet. La version du système d'exploitation android est également différente entre les appareils : de la 4.0 Ice Cream Sandwich à la 5.0 Lollipop. Ceci m'assure de toucher un public large, mon application étant compatible depuis l'API 14, autrement dit 4.0, 4.0.1, 4.0.2 Ice Cream Sandwich. Par ailleurs, afin d'assurer la portabilité de mon travail entre les différents ordinateurs, j'ai créé des repositories sur GitHub, aussi bien pour le code que pour le rapport. Ceci, mis à part la portabilité, me procure aussi un backup supplémentaire online.\\

Enfin, j'utilise divers outils, pas forcément technologiques, afin de compléter mon travail de fin d'études. Ceux-ci sont :
\begin{itemize}
\item Adobe Illustrator pour les graphismes, beaucoup plus adaptables lorsqu'ils sont réalisés en dessin vectoriel.
\item OpenDyslexic, une police de caractère adaptée pour faciliter la lecture, sur le bon conseils de la logopède, Laurence Henrion.
\item le VOB (Vocabulaire Orthographique de Base) du premier degré, pour les mots que les enfants devront lire ou reconstituer lors des jeux.
\end{itemize} 