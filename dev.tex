\section{Développement de l'application}
Ce point du rapport est dédié au déroulement du développement de l'application \textit{Manabu}. J'expliquerai ce que j'ai mis en place point par point, en fonction des différentes parties de l'application. Je commencerai par quelques généralités concernant le programme, puis j'expliquerai le menu principal, et enfin les différents exercices. J'illustrerai au maximum mes propos par des screenshots des différents exercices.

\subsection{Généralités}
Ce que j'appelle généralités sont pour moi des décisions que j'ai prises et qui affectent l'application dans son ensemble, où simplement des points nécessitant d'être explicités pour une meilleure compréhension du développement de l'application.\\

Tout d'abord, avant de me lancer dans la création de \textit{Manabu}, j'ai suivi le premier tutoriel disponible sur le site officiel d'Android pour le développeurs, \textit{Building Your First App}\footnote{Building Your First App, \url{http://developer.android.com/training/basics/firstapp/index.html}, consulté en janvier et février 2015}. Ceci m'a permis de me familiariser avec le SDK Android et de connaître les bases de ce type de programmation.\\

Ensuite, il faut savoir que \textit{Manabu} est composée de 5 activités. Une activité, comme expliqué au point \textit{Outils et technologies utilisés} dans la définition d'Android Developer, représente un écran avec lequel l'utilisateur peut interagir. Dans la pratique, une activité est contenue dans une classe java qui lui est propre et associée avec un où plusieurs layouts. Dans le cas présent, j'ai considéré que mon application est constituée de 5 parties différentes. J'ai donc transcrit chacune d'entre elles en une activité. Celles-ci sont :
\begin{itemize}
\item la \textit{main activity}, autrement dit le menu principal. Il s'agit de l'écran d'ouverture de l'application.
\item l'activité de l'exercice \textit{Imagerie}.
\item l'activité de l'exercice \textit{Lecture flash}.
\item l'activité de l'exercice \textit{Anagrammes}.
\item l'activité de l'exercice \textit{Écouter le son}.\\
\end{itemize}

L'activité contenant le menu principal ne contient qu'un seul layout. Les autres activités, celles des exercices, en contiennent quant à elles plusieurs. Trois layouts sont communs au quatre exercices. Il s'agit du layout de menu de démarrage, contenant notamment le choix du niveau, du layout d'affichage des règles du jeu et du layout de fin d'exercice.\\

Par ailleurs, j'ai intégré \textit{OpenDyslexic} en tant que police principale de \textit{Manabu}. Ça n'a pas été la tâche la plus simple. J'ai tout d'abord essayé de modifier la police d'un élément (pensant par la suite pouvoir l'appliquer à toute l'application) via la méthode "classique". C'est-à-dire aller chercher la police dans un dossier et l'appliquer à chaque élément voulu. Cette méthode s'est révélée lourde à mettre en place. C'est en voulant intégrer par la suite la librairie \textit{Calligraphy} de ChrisJenx\footnote{Disponible gratuitement sur GitHub à l'adresse \url{https://github.com/chrisjenx/Calligraphy}.} permettant d'utiliser une police pour tout l'application facilement que j'ai rencontré le problème avec Maven dans Eclipse. J'ai donc, comme dit précédemment, changé d'IDE pour Android Studio. Gradle a quant à lui ajouté sans problème la librairie à la liste des dépendances.

\subsection{Le menu principal}
Le menu principal représente l'activité de base de l'application, celle depuis laquelle peuvent être démarrées les autres activités contenant les exercices. Le layout formant ce menu est relativement simple. Il s'agit de quatre boutons disposés en rectangle. Chacun des boutons lance bien évidemment un des exercices.\\

Pour chaque bouton, il m'a fallu implémenter une fonction spécialisée permettant de lancer la nouvelle activité correspondant à l'exercice désiré.
Lancer une nouvelle activité se fait à l'aide d'un \textit{intent}. Un intent est une sorte de lien de communication pour le système Android. Celui-ci signale qu'il se passe quelque chose. Dans le cas présent, l'intent signale que l'activité du menu principal va en démarrer une autre.\\

Enfin, je tiens à mentionner qu'il n'est pas possible de lancer les activités de plusieurs exercices à la fois. En effet, lorsqu'un exercice est lancé depuis le menu, l'activité en cours représentée passe du menu à l'activité de l'exercice, l'activité menu étant alors en attente. Il faut quitter l'exercice, et donc l'activité, pour revenir au menu et pouvoir en démarrer une autre.

%La toute première activité que j'ai mise en place a été le squelette de l'application. Par squelette, je veux dire le menu de base, ainsi que la manière dont chaque exercice serait lancé, et le menu présent pour chaque exercice (semblable pour tous). \\

%Mettre en place le layout du menu principal n'a pas été trop difficile, celui-ci étant simplement composé de quatre boutons disposés en rectangle, ceux-ci lançant une nouvelle \textit{activity} pour chaque exercice. La tâche s'est compliquée lorsque j'ai voulu faire en sorte que la police principale de l'application devienne \textit{OpenDyslexic}, comme me l'avait conseillé la logopède. J'ai tout d'abord essayé de modifier la police d'un élément (pensant par la suite pouvoir l'appliquer à toute l'application sous forme de fonction) via la méthode "classique". C'est-à-dire aller chercher la police dans un dossier et l'appliquer à chaque élément voulu. Cette méthode n'a pas fonctionné, et s'est révélée être lourde à mettre en place. Je me suis alors tournée vers une librairie externe : \textit{Calligraphy} de ChrisJenx\footnote{Disponible gratuitement sur GitHub à l'adresse \url{https://github.com/chrisjenx/Calligraphy}.}.\\

%Le mode d'emploi d'ajout de \textit{Calligraphy} étant clair, j'ai décidé d'intégrer cette librairie via Maven dans mon projet Eclipse. Pour ce faire, il me fallait installer le plugin Maven et convertir ensuite mon projet. J'ai à ce stade du travail rencontré le problème avec Eclipse décrit précédemment. De ce fait, j'ai changé mon fusil d'épaule pour continuer mon projet avec Android Studio. Ce dernier utilisant Gradle, plus facile d'utilisation que Maven, il m'a suffit d'ajouter une ligne dans le fichier spécifiant les dépendances pour importer la librairie. J'ai alors pu utiliser la police voulue pour mon application en suivant les instructions fournies par le créateur de la librairie.\\



\subsection{L'exercice \textit{Imagerie}}
L'exercice \textit{Imagerie} est le premier que j'ai choisi d'implémenter. Il s'agit de mon exercice favori, car il m'a permis d'exprimer ma créativité au travers des dessins réalisés. Pour le premier niveau, j'ai choisi de partir sur la version la plus simple des fichiers freinet : une image affichée avec un mot à mémorisé, et puis sur base de la même image retrouver ce mot parmi trois propositions.\\

J'ai commencé en premier lieu par définir une liste de vingt mots à illustrer pour le niveau 1. Pour chacun de ces mots, j'ai également choisi deux autres mots ressemblants au premier pour faire office de fausses réponses. A partir de cette liste, j'ai illustré les mots corrects à l'aide d'Illustrator. Je n'ai pas tout dessiné d'une traite. En parallèle du dessin, j'ai mis en place l'algorithme de l'exercice.\\

Pour la partie programmation, j'ai débuté en définissant les layouts au format xml. Il me fallait un layout pour l'affichage de l'image et du mot correct, et un autre pour l'affichage de l'image et le choix des réponses. Je préfère travailler avec plusieurs layouts plutôt que de me compliquer la tâche à ajouter ou retirer des éléments à l'aide de lignes de code (Java), ce qui me paraît plus lourd.\\

 Une fois les layouts définis, la première étape a été d'afficher une image précise avec son mot, de la valider, puis de passer au choix multiple pour cette image. Dans l'arborescence d'Android, les images sont stockées dans des répertoires appelés \textit{drawable}, et les chaînes de caractères dans un fichier nommé \textit{strings.xml}. Sachant cela, et afin de me faciliter la tâche, j'ai donné à l'image et au mot le même nom. Ce nom est de type \textit{img\_XX} où \textit{XX} correspond au numéro identifiant l'image. Dans le cas du niveau 1, j'ai pour ce TFE créé 21 images. Les numéros s'étendent donc de 0 à 20. Pour les deux réponses incorrectes, je les ai respectivement nommées \textit{img\_XX\_1} et \textit{img\_XX\_2}. Le choix de cette nomenclature s'expliquera dans le paragraphe suivant. En ce qui concerne l'étape dont je parle actuellement, elle a été réalisée avec une seule image, hardcodée.\\
 
Par après, j'ai rajouté le choix de l'image au hasard, ainsi que l'affichage de l'ordre des réponses au hasard. C'est ici que le choix de la nomenclature prend tout son sens. En effet, le \textit{XX} précédemment cité. Celui-ci est choisi au hasard dans l'intervalle spécifié, pour ensuite être concaténé afin de former la chaîne de caractère correspondant aux identifiants de l'image et du mot. Enfin, j'ai implémenté une boucle afin que l'exercice soit une série de 10, tout en m'assurant que les mots ne puissent pas être deux fois identiques dans cette série en mémorisant ceux piochés précédemment.\\

Concernant les réponses proposées pour l'exercice, il m'a fallu trouvé un système pour que l'enfant qui joue sache clairement s'il s'est trompé où s'il a réussi. J'ai pour ce faire mis en place un \textit{toast} qui apparaît lorsqu'on clique sur un des boutons de réponse. En Android, un \textit{toast} est une sorte de notification qui se surimprime sur l'écran pendant un temps défini. Le plus souvent, il s'agit d'un message simple. Parfois, on y retrouve une image, pour ce faire, un layout est créé pour le \textit{toast}. Je voulais mettre en place un toast avec une image et un texte : "V" et "Bien joué !" pour la bonne réponse, "X" et "Essaye encore !" pour la mauvaise. J'ai donc créé un layout spécifique, utilisable dans les deux cas, car je passe l'identifiant de l'image et le message en paramètre à l'aide d'une fonction. J'ai fait de ce \textit{toast} une fonction utilisable dans les différentes \textit{activities} qui composent mon application.\\

L'exercice \textit{Imagerie} n'a pas été trop difficile à mettre en place. Une fois la logique définie, j'ai aisément pu implémenter les fonctions nécessaires. Toutefois, il s'agissait du premier exercice, j'ai donc du appréhender certaines notions. J'ai par exemple appris à lier une image ou une chaîne de caractères à un élément du layout xml à l'aide du code java afin de pouvoir le modifier.

\subsection{L'exercice \textit{Lecture flash}}
L'exercice \textit{Lecture flash} est le deuxième que j'ai mis en place pour l'application. Comme pour l'exercice précédent, j'ai d'abord commencé par le premier niveau de difficulté. Dans le cas présent, la difficulté entre les niveaux se situe principalement au niveau de la vitesse de lecture. Pour ce premier niveau, j'ai choisi de laisser la possibilité d'afficher le mot pendant 20 secondes.\\

Les mots utilisés pour cet exercices n'ont pas été choisis au hasard. En effet, suivant les conseils de la logopède, Laurence Henrion, j'utilise le VOB (Vocabulaire Orthographique de Base). Pour rappel, il s'agit d'une liste de vocabulaire que les enfants doivent maîtriser à la fin de chaque cycle. Dans le cadre de l'application, j'utilise le VOB du degré inférieur, qui correspond aux mots devant être connus fin de deuxième primaire (cf. annexe \ref{listeVob}).\\

La première étape de programmation de cet exercice était donc de recopier le 480 mots constituant le VOB du cycle inférieur dans le fichier \textit{strings.xml}. Comme précédemment, afin de faciliter le choix des mots de manière aléatoire, les noms sont identiques et différenciés par un nombre. La nomenclature de ceux-ci est \textit{str\_XXX}. \\

Une fois les mots recopiés, tout comme pour l'exercice précédent, j'ai défini les layouts \textit{xml}. Dans le cas présent, il me fallait 3 layouts :
\begin{itemize}
\item un premier lors du démarrage, afin de choisir le nombre de secondes d'affichage des mots.
\item un deuxième pour l'affichage du mot en lui-même. Très simple car il est constitué d'un seul élément.
\item Un troisième et dernier avec un champ texte éditable et un bouton de vérification, qui est chargé une fois que le temps d'affichage du mot est écoulé.
\end{itemize}
Ces layouts ne sont pas composés de beaucoup d'éléments, ce qui m'a permis de réaliser cette étape assez rapidement.\\

Du point de vue de la programmation Java, j'ai tout d'abord créé un \textit{NumberPicker} personnalisé pour définir le temps d'affichage du mot. Un \textit{NumberPicker} est un élément que l'on peut ajouter tel quel à un layout et qui permet de sélectionner un nombre dans un intervalle. Or, l'élément en tant que tel est très peu personnalisable. De ce fait, j'ai choisi de mettre en place le mien, ce qui est facile à faire. J'ai simplement aligné deux boutons avec un champ texte non éditable entre eux. J'ai assigné au bouton "-" la décrémentation du champ texte, et au "+" l'incrémentation.\\

La suite de la programmation s'est déroulée de manière fluide également : l'affichage du mot le temps voulu (celui-ci récupéré du \textit{NumberPicker} codé au layout précédent), le choix au hasard du mot parmi le VOB, et la vérification du mot post-lecture. Pour le choix du mot au hasard, j'ai réutilisé le code de l'exercice \textit{Imagerie} et le modifiant pour qu'il corresponde à l'exercice.\\

A ce stade, l'exercice en lui-même était fonctionnel. J'y suis revenue par la suite afin d'implémenter un clavier propre à \textit{Manabu}. Ceci m'a été demandé par les enfants sur lequels j'ai eu l'occasion de tester l'application, et notamment cet exercice (cf. point \ref{testFlash}). Ceux-ci préféraient avoir un clavier pour lequel il ne devaient pas réapprendre l'ordre des lettres, et donc un de type \textit{alphabet} plutôt qu'un \textit{azerty}. Afin de mettre en place mon propre \textit{SoftKeyboard}, je me suis inspirée du tutoriel de Martin Pennings et j'ai téléchargé le code source disponible sur la page du celui-ci\footnote{Maarten Pennings,\textit{Android development: Custom keyboard}, \url{http://www.fampennings.nl/maarten/android/09keyboard/index.htm}, consulté le 18 mai 2015}. Après avoir essayer de compléter mon code sur base du tutoriel seul, sans grand succès, j'ai décidé d'intégrer le code source précédemment téléchargé à mon projet.\\

Enfin, le code source de Martin Pennings étant pour mettre en place un \textit{Softkeyboard} hexadécimal, je ne l'ai pas gardé tel quel. J'ai remplacé le seul layout fourni de base par quatre nouveau layouts composants mon clavier :
\begin{itemize}
\item un layout avec les 26 lettres de l'alphabet en minucule
\item un layout avec les 26 lettres de l'alphabet en majuscule
\item un layout avec les lettres accentuées en minuscule et la ponctuation
\item un layout avec les lettres accentuées en majuscule et la ponctuation
\end{itemize}
De ce fait, j'ai également modifié certaines parties du code précédemment intégré afin de l'adapter aux besoins de Manabu.\\

La mise en place du premier niveau de l'exercice \textit{Lecture Flash} en lui-même s'est donc déroulée sans encombre. Comme expliqué ci-dessus, la partie la plus ardue a été l'implémentation du clavier \textit{alphabet} à partir du code de quelqu'un d'autre.
	
\subsection{L'exercice \textit{Anagrammes}}
\textit{Anagrammes} est le troisième exercice que j'ai implémenté. C'est également à ce stade du développement, après avoir commencé à mettre en place le mélange des lettres des mots, que j'ai effectué le test de l'application sur des enfants (cf. \ref{testEnfants}).

\subsection{L'exercice \textit{Ecouter le son}}
