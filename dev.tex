\section{Déroulement du développement}
- Mise en place du squelette de l'appli. Départ sur Eclipse, migration vers Android Studio pour cause d'incompétence et de bugs répétitifs sur Eclipse Juno.

\subsection{L'exercice \textit{Imagerie}}
\subsection{L'exercice \textit{Lecture flash}}
\subsection{L'exercice \textit{Anagrammes}}
\subsection{L'exercice \textit{Ecouter le son}}
\subsection{Le test sur les enfants}
Durant le développement de Manabu, j'ai eu l'occasion d'aller tester l'application sur des enfants de première primaire. Cette occasion s'est présentée à moi grâce à Mme Van den Schrieck, qui m'a donné les coordonnées d'une dame, Mme Aliette Lochy, réalisant des tests concernant la lecture et la reconnaissance des visage sur des enfants de première primaire durant les vacances de Pâques. J'ai donc pris contact avec Mme Lochy pour lui demander s'il était possible de participer à une des séances organisées avec les enfants afin d'avoir un feedback sur mon application, et pouvoir éventuellement rediriger le tir. Je me suis donc rendue à la faculté de psychologie de Louvain-La-Neuve le jeudi 9 avril, après accord de Mme Lochy.\\

Quatre enfants de première primaire (deux filles et deux garçons) étaient à ma disposition ce jour-là, afin que l'on parcoure ensemble les exercices déjà mis en place et qu'ils me donnent leur avis sur chacun d'entre eux. Le test a été réalisé sur la tablette Samsung Galaxy Tab 2 10.1, plus conviviale qu'un smartphone. Les exercices déjà développés dans l'application pour être suffisamment fonctionnels ce jour là étaient :
\begin{itemize}
\item Imagerie : niveau 1
\item Lecture flash : niveau 1
\item Anagrammes : niveau 1, mais incomplet.
\end{itemize}
Je ne compte pas détailler ici ce que m'a dit chaque enfant, les commentaires étant très similaires pour chacun d'entre eux. Je vais expliquer ma procédure de test, un exercice à la fois, et détailler par exercice les questions posées aux enfants ainsi que les réponses obtenues.

\subsubsection{Test 1 : l'exercice \textit{Imagerie}}
Pour cet exercice, j'ai tout d'abord expliqué aux enfants les règles du jeu. Il s'agissait, pour une série de 10 images, de lire le mot associé à chacune d'entre elle. Ils appuyaient ensuite sur le bouton \textit{mémorisé}, et devaient enfin choisir parmi les 3 choix proposés lequel correspondait au mot lu précédemment, l'image étant toujours affichée.

\subsubsection{Test 2 : l'exercice \textit{Lecture flash}}
Comme précedemment, j'ai commencé par expliquer aux enfants les règles du jeu.

\subsubsection{Test 3 : l'exercice \textit{Anagrammes}}


\subsubsection{Points relevés}