\section{Déroulement du développement}
Cette partie du rapport est dédiée au déroulement du développement de l'application \textit{Manabu}. Je l'explique point par point, en fonction des différentes parties de l'application, de manière chronologique\footnote{Il se peut qu'il y ait certains sauts dans le temps, des ellipses, dues à un changement d'exercice en cours de développement de ma part}. Ce point est également dédié au test de \textit{Manabu} que j'ai effectué sur des enfants en fin de première primaire dans le courant du mois d'avril. En effet, ce test au milieu de la création de l'application m'a permis de confirmer certaines décisions que je n'étais pas sûre de prendre (telle la présence d'une musique de fond ou non, cf. plus bas, point \ref{testPoints}). 

\subsection{Le menu principal}
La toute première chose que j'ai mise en place a été le squelette de l'application. Par squelette, je veux dire le menu de base, ainsi que la manière dont chaque exercice serait lancé, et le menu présent pour chaque exercice (semblable pour tous). \\

Comme mentionné dans le point précédent, \textit{Outils et technologies utilisées}, j'ai commencé le développement sur Eclipe Juno. Toutefois, avant de me lancer dans la création de \textit{Manabu}, j'ai suivi le premier tutoriel disponible sur le site officiel d'Android pour le développeurs, \textit{Building Your First App}\footnote{Building Your First App, \url{http://developer.android.com/training/basics/firstapp/index.html}, consulté en janvier et février 2015}. Ceci m'a permis de me familiariser avec le SDK Android et de connaître les bases de ce type de programmation.\\

Mettre en place le layout du menu principal n'a pas été trop difficile, celui-ci étant simplement composé de quatre boutons disposés en rectangle, ceux-ci lançant une nouvelle \textit{activity} pour chaque exercice. La tâche s'est compliquée lorsque j'ai voulu faire en sorte que la police principale de l'application devienne \textit{OpenDyslexic}, comme me l'avait conseillé la logopède. J'ai tout d'abord essayé de modifier la police d'un élément (pensant par la suite pouvoir l'appliquer à toute l'application sous forme de fonction) via la méthode "classique". C'est-à-dire aller chercher la police dans un dossier et l'appliquer à chaque élément voulu. Cette méthode n'a pas fonctionné, et s'est révélée être lourde à mettre en place. Je me suis alors tournée vers une librairie externe : \textit{Calligraphy} de ChrisJenx\footnote{Disponible gratuitement sur GitHub à l'adresse \url{https://github.com/chrisjenx/Calligraphy}.}.\\

Le mode d'emploi d'ajout de \textit{Calligraphy} étant clair, j'ai décidé d'intégrer cette librairie via Maven dans mon projet Eclipse. Pour ce faire, il me fallait installer le plugin Maven et convertir ensuite mon projet. J'ai à ce stade du travail rencontré le problème avec Eclipse décrit précédemment. De ce fait, j'ai changé mon fusil d'épaule pour continuer mon projet avec Android Studio. Ce dernier utilisant Gradle, plus facile d'utilisation que Maven, il m'a suffit d'ajouter une ligne dans le fichier spécifiant les dépendances pour importer la librairie. J'ai alors pu utiliser la police voulue pour mon application en suivant les instructions fournies par le créateur de la librairie.\\

Une fois le menu principal finalisé, j'ai implémenté les fonctions permettant de lancer une activité par jeu, ainsi que le second menu s'affichant pour chaque exercice. Ce menu est organisé de la manière suivante :
\begin{itemize}
\item Etoiles pour choisir le niveau (non implémentées directement, venues se rajouter par la suite)
\item Affichage des règles du jeu
\item Démarrer le jeu
\end{itemize}
Bien que les exercices soient tous différents, j'ai créé par soucis d'optimisation un layout unique et pour le menu, et pour les règles du jeu, ainsi que des fonctions communes pour ceux-ci.


\subsection{L'exercice \textit{Imagerie}}
L'exercice \textit{Imagerie} est le premier que j'ai choisi d'implémenter. Il s'agit de mon exercice favori, car il m'a permis d'exprimer ma créativité au travers des dessins réalisés. Pour le premier niveau, j'ai choisi de partir sur la version la plus simple des fichiers freinet : une image affichée avec un mot à mémorisé, et puis sur base de la même image retrouver ce mot parmi trois propositions.\\

J'ai commencé en premier lieu par définir une liste de vingt mots à illustrer pour le niveau 1. Pour chacun de ces mots, j'ai également choisi deux autres mots ressemblants au premier pour faire office de fausses réponses. A partir de cette liste, j'ai illustré les mots corrects à l'aide d'Illustrator. Je n'ai pas tout dessiné d'une traite. En parallèle du dessin, j'ai mis en place l'algorithme de l'exercice.\\

Pour la partie programmation, j'ai débuté en définissant les layouts au format xml. Il me fallait un layout pour l'affichage de l'image et du mot correct, et un autre pour l'affichage de l'image et le choix des réponses. Je préfère travailler avec plusieurs layouts plutôt que de me compliquer la tâche à ajouter ou retirer des éléments à l'aide de lignes de code (Java), ce qui me paraît plus lourd.\\

 Une fois les layouts définis, la première étape a été d'afficher une image précise avec son mot, de la valider, puis de passer au choix multiple pour cette image. Dans l'arborescence d'Android, les images sont stockées dans des répertoires appelés \textit{drawable}, et les chaînes de caractères dans un fichier nommé \textit{strings.xml}. Sachant cela, et afin de me faciliter la tâche, j'ai donné à l'image et au mot le même nom. Ce nom est de type \textit{img\_XX} où \textit{XX} correspond au numéro identifiant l'image. Dans le cas du niveau 1, j'ai pour ce TFE créé 21 images. Les numéros s'étendent donc de 0 à 20. Pour les deux réponses incorrectes, je les ai respectivement nommées \textit{img\_XX\_1} et \textit{img\_XX\_2}. Le choix de cette nomenclature s'expliquera dans le paragraphe suivant. En ce qui concerne l'étape dont je parle actuellement, elle a été réalisée avec une seule image, hardcodée.\\
 
Par après, j'ai rajouté le choix de l'image au hasard, ainsi que l'affichage de l'ordre des réponses au hasard. C'est ici que le choix de la nomenclature prend tout son sens. En effet, le \textit{XX} précédemment cité. Celui-ci est choisi au hasard dans l'intervalle spécifié, pour ensuite être concaténé afin de former la chaîne de caractère correspondant aux identifiants de l'image et du mot. Enfin, j'ai implémenté une boucle afin que l'exercice soit une série de 10, tout en m'assurant que les mots ne puissent pas être deux fois identiques dans cette série en mémorisant ceux piochés précédemment.\\

Concernant les réponses proposées pour l'exercice, il m'a fallu trouvé un système pour que l'enfant qui joue sache clairement s'il s'est trompé où s'il a réussi. J'ai pour ce faire mis en place un \textit{toast} qui apparaît lorsqu'on clique sur un des boutons de réponse. En Android, un \textit{toast} est une sorte de notification qui se surimprime sur l'écran pendant un temps défini. Le plus souvent, il s'agit d'un message simple. Parfois, on y retrouve une image, pour ce faire, un layout est créé pour le \textit{toast}. Je voulais mettre en place un toast avec une image et un texte : "V" et "Bien joué !" pour la bonne réponse, "X" et "Essaye encore !" pour la mauvaise. J'ai donc créé un layout spécifique, utilisable dans les deux cas, car je passe l'identifiant de l'image et le message en paramètre à l'aide d'une fonction. J'ai fait de ce \textit{toast} une fonction utilisable dans les différentes \textit{activities} qui composent mon application.\\

L'exercice \textit{Imagerie} n'a pas été trop difficile à mettre en place. Une fois la logique définie, j'ai aisément pu implémenter les fonctions nécessaires. Toutefois, il s'agissait du premier exercice, j'ai donc du appréhender certaines notions. J'ai par exemple appris à lier une image ou une chaîne de caractères à un élément du layout xml à l'aide du code java afin de pouvoir le modifier.

\subsection{L'exercice \textit{Lecture flash}}
L'exercice \textit{Lecture flash} est le deuxième que j'ai mis en place pour l'application. Comme pour l'exercice précédent, j'ai d'abord commencé par le premier niveau de difficulté. Dans le cas présent, la difficulté entre les niveaux se situe principalement au niveau de la vitesse de lecture. Pour ce premier niveau, j'ai choisi de laisser la possibilité d'afficher le mot pendant 20 secondes.\\

Les mots utilisés pour cet exercices n'ont pas été choisis au hasard. En effet, suivant les conseils de la logopède, Laurence Henrion, j'utilise le VOB (Vocabulaire Orthographique de Base). Pour rappel, il s'agit d'une liste de vocabulaire que les enfants doivent maîtriser à la fin de chaque cycle. Dans le cadre de l'application, j'utilise le VOB du degré inférieur, qui correspond aux mots devant être connus fin de deuxième primaire (cf. annexe \ref{listeVob}).\\

La première étape de programmation de cet exercice était donc de recopier le 480 mots constituant le VOB du cycle inférieur dans le fichier \textit{strings.xml}. Comme précédemment, afin de faciliter le choix des mots de manière aléatoire, les noms sont identiques et différenciés par un nombre. La nomenclature de ceux-ci est \textit{str\_XXX}. \\

Une fois les mots recopiés, tout comme pour l'exercice précédent, j'ai défini les layouts \textit{xml}. Dans le cas présent, il me fallait 3 layouts :
\begin{itemize}
\item un premier lors du démarrage, afin de choisir le nombre de secondes d'affichage des mots.
\item un deuxième pour l'affichage du mot en lui-même. Très simple car il est constitué d'un seul élément.
\item Un troisième et dernier avec un champ texte éditable et un bouton de vérification, qui est chargé une fois que le temps d'affichage du mot est écoulé.
\end{itemize}
Ces layouts ne sont pas composés de beaucoup d'éléments, ce qui m'a permis de réaliser cette étape assez rapidement.\\

Du point de vue de la programmation Java, j'ai tout d'abord créé un \textit{NumberPicker} personnalisé pour définir le temps d'affichage du mot. Un \textit{NumberPicker} est un élément que l'on peut ajouter tel quel à un layout et qui permet de sélectionner un nombre dans un intervalle. Or, l'élément en tant que tel est très peu personnalisable. De ce fait, j'ai choisi de mettre en place le mien, ce qui est facile à faire. J'ai simplement aligné deux boutons avec un champ texte non éditable entre eux. J'ai assigné au bouton "-" la décrémentation du champ texte, et au "+" l'incrémentation.\\

La suite de la programmation s'est déroulée de manière fluide également : l'affichage du mot le temps voulu (celui-ci récupéré du \textit{NumberPicker} codé au layout précédent), le choix au hasard du mot parmi le VOB, et la vérification du mot post-lecture. Pour le choix du mot au hasard, j'ai réutilisé le code de l'exercice \textit{Imagerie} et le modifiant pour qu'il corresponde à l'exercice.\\

A ce stade, l'exercice en lui-même était fonctionnel. J'y suis revenue par la suite afin d'implémenter un clavier propre à \textit{Manabu}. Ceci m'a été demandé par les enfants sur lequels j'ai eu l'occasion de tester l'application, et notamment cet exercice (cf. point \ref{testFlash}). Ceux-ci préféraient avoir un clavier pour lequel il ne devaient pas réapprendre l'ordre des lettres, et donc un de type \textit{alphabet} plutôt qu'un \textit{azerty}. Afin de mettre en place mon propre \textit{SoftKeyboard}, je me suis inspirée du tutoriel de Martin Pennings et j'ai téléchargé le code source disponible sur la page du celui-ci\footnote{Maarten Pennings,\textit{Android development: Custom keyboard}, \url{http://www.fampennings.nl/maarten/android/09keyboard/index.htm}, consulté le 18 mai 2015}. Après avoir essayer de compléter mon code sur base du tutoriel seul, sans grand succès, j'ai décidé d'intégrer le code source précédemment téléchargé à mon projet.\\

Enfin, le code source de Martin Pennings étant pour mettre en place un \textit{Softkeyboard} hexadécimal, je ne l'ai pas gardé tel quel. J'ai remplacé le seul layout fourni de base par quatre nouveau layouts composants mon clavier :
\begin{itemize}
\item un layout avec les 26 lettres de l'alphabet en minucule
\item un layout avec les 26 lettres de l'alphabet en majuscule
\item un layout avec les lettres accentuées en minuscule et la ponctuation
\item un layout avec les lettres accentuées en majuscule et la ponctuation
\end{itemize}
De ce fait, j'ai également modifié certaines parties du code précédemment intégré afin de l'adapter aux besoins de Manabu.\\

La mise en place du premier niveau de l'exercice \textit{Lecture Flash} en lui-même s'est donc déroulée sans encombre. Comme expliqué ci-dessus, la partie la plus ardue a été l'implémentation du clavier \textit{alphabet} à partir du code de quelqu'un d'autre.
	
\subsection{L'exercice \textit{Anagrammes}}
\textit{Anagrammes} est le troisième exercice que j'ai implémenté. C'est également à ce stade du développement, après avoir commencé à mettre en place le mélange des lettres des mots, que j'ai effectué le test de l'application sur des enfants (cf. \ref{testEnfants}).

\subsection{L'exercice \textit{Ecouter le son}}


\subsection{Le test sur les enfants}
Durant le développement de Manabu, j'ai eu l'occasion d'aller tester l'application sur des enfants de première primaire. Cette occasion s'est présentée à moi grâce à Mme Van den Schrieck, qui m'a donné les coordonnées d'une dame, Mme Aliette Lochy, réalisant des tests concernant la lecture et la reconnaissance des visage sur des enfants de première primaire durant les vacances de Pâques. J'ai donc pris contact avec Mme Lochy pour lui demander s'il était possible de participer à une des séances organisées avec les enfants afin d'avoir un feedback sur mon application, et pouvoir éventuellement rediriger le tir. Je me suis donc rendue à la faculté de psychologie de Louvain-La-Neuve le jeudi 9 avril, après accord de Mme Lochy.\\

Quatre enfants de première primaire (deux filles et deux garçons) étaient à ma disposition ce jour-là, afin que l'on parcoure ensemble les exercices déjà mis en place et qu'ils me donnent leur avis sur chacun d'entre eux. Le test a été réalisé sur la tablette Samsung Galaxy Tab 2 10.1, plus conviviale qu'un smartphone. Les exercices déjà développés dans l'application pour être suffisamment fonctionnels ce jour là étaient :
\begin{itemize}
\item Imagerie : niveau 1
\item Lecture flash : niveau 1
\item Anagrammes : niveau 1, mais incomplet.
\end{itemize}
Je ne compte pas détailler ici ce que m'a dit chaque enfant, les commentaires étant très similaires pour chacun d'entre eux. Je vais expliquer ma procédure de test, un exercice à la fois, et détailler par exercice les questions posées aux enfants ainsi que les réponses obtenues.

\subsubsection{Test 1 : l'exercice \textit{Imagerie}}
Pour cet exercice, j'ai tout d'abord expliqué aux enfants les règles du jeu. Il s'agissait, pour une série de 10 images, de lire le mot associé à chacune d'entre elle. Par image, ils appuyaient ensuite sur le bouton \textit{mémorisé}, et devaient enfin choisir parmi les 3 choix proposés lequel correspondait au mot lu précédemment, l'image étant toujours affichée.\\

Pour rappel, lors de ce test, le seul niveau implémenté était le niveau 1 : un mot = une image.\\

Les commentaires que j'ai obtenu de la part des enfants étaient similaires. Ils m'ont tous mentionné que le niveau de cet exercice était trop facile pour leur niveau d'apprentissage (quasi fin de première primaire). Lorsque je leur ai posé la question de ce qui pourrait être plus de leur niveau en demandant si une phrase complète plutôt qu'un mot serait plus difficile, ils m'ont répondu positivement. J'en ai déduit qu'il faut soit des exercices plus compliqués, soit une discrimination. Par discrimination, j'entends, montrer deux images associées chacune à un mot ou une phrase, et pour le choix, montrer une troisième image avec le mot correspondant à sélectionner plutôt que les deux montrés précédemment..

\subsubsection{Test 2 : l'exercice \textit{Lecture flash}}
Comme précédemment, j'ai commencé par expliquer aux enfants les règles du jeu.
Ils devaient commencer par me dire environ combien de secondes ils avaient besoin pour lire un mot. Pour ce faire, je leur permettait de tester différents timings afin de choisir celui qui leur convenait le mieux. Une fois le timing choisi, je leur expliquais le principe. Pour une série de 10 mots, ils devaient lire le mot dans le temps imparti, et ensuite ré-écrire celui-ci à l'aide du clavier de la tablette, accent et caractères spéciaux compris.\\

Le temps de lecture était variable selon les enfants. Les filles étaient plus rapides que les garçons pour lire un mot de taille moyenne (plus de 4 ou 5 lettres). J'avais donc, pour le niveau 1, une durée de lecture variant entre 10 et 20 secondes selon les enfants.\\

Cet exercice de lecture flash était considéré comme plus difficile par les enfants. En effet, ils avaient besoin de plus de concentration, car il leur fallait retenir le mot pour pouvoir le ré-écrire. Un des garçons m'a cependant dit qu'il trouvait cet exercice facile. Pourtant, au vu de ses résultats, j'ai constaté qu'il préférait écrire le mot plutôt que de le lire, ce qui lui posait quelques problèmes.\\

Enfin, la ré-écriture du mot posait problème. Non pas que les enfants n'avaient pas retenu le mot, mais le clavier par défaut en AZERTY les perturbait. Ils rencontraient des difficultés à situer les lettres sur le clavier, et confondaient certaines d'entre elles (\textit{b} et \textit{d}, \textit{q} et \textit{p}). De ce fait, ils m'ont tous mentionné qu'il préféreraient disposer d'un clavier de type \textit{alphabet}, avec les accents et caractères type \textit{-, ', ...} à disposition.

\subsubsection{Test 3 : l'exercice \textit{Anagrammes}}
Tel que mentionné auparavant, cet exercice n'était pas complet lors de son évaluation auprès des enfants. J'avais implémenté l'algorithme de création de l'anagramme, mais pas sa validation une fois les lettres remises dans l'ordre. Par ailleurs, l'affichage à ce moment présentait une zone où le mot était écrit, à côté des lettres mélangées.\\

J'ai tout d'abord demandé aux enfants de remettre les lettres dans l'ordre, sur base du mot qu'ils lisaient juste à côté. Ceci était bien entendu trop facile pour eux. Ils n'avaient qu'à regarder les lettres du mot, et à copier.\\

Ensuite, j'ai caché le mot affiché en entier pour ne laisser que les lettres mélangées, et j'ai demandé de remettre les lettres dans l'ordre pour former un mot. Les mots étaient constitués au maximum de 5 lettres. Sans aucune aide extérieure, il était quasi impossible pour les enfants de remettre les lettres en ordre. J'ai alors essayé le même type d'exercice, mais en prononçant le mot. Dès lors, les enfants réussissaient l'exercice, à l'exception de quelques mots.\\

Les mots posant problème aux enfants étaient ceux qui comprennent plusieurs phonèmes similaires, mais ne s'écrivant pas de la même manière. Ainsi, pour le mot "aimer" par exemple, un des enfants cherchait deux fois la lettre \textit{e} pour écrire le son "é" entendu au début et à la fin du mot. J'ai du lui donner un indice en lui expliquant l'association des voyelles pour former un son afin qu'il trouve l'orthographe correcte.\\

Ces essais concernant les anagrammes m'ont permis de clarifier la méthode à employer pour mettre en place l'exercice. J'en ai déduit qu'il était plus simple pour l'enfant d'associer le mot et l'ordre des lettres à partir du son du mot prononcé. J'ai donc choisi de mettre jouer le mot de manière sonore lors de la génération de l'anagramme et de laisser la possibilité à l'enfant de rejouer celui-ci afin de l'aider.

\subsubsection{Points relevés \label{testPoints}}
Parmi les questions posées et les avis obtenus de la part des enfants, voici les principaux points qui peuvent être relevés concernant les exercices :
\begin{itemize}
\item Fin de première primaire, les enfants ont le niveau suffisament pour savoir lire plus ou moins aisément un mot seul. A ce stade, il est plus intéressant de s'orienter vers les phrases ou la discrimination des mots. Néanmoins, les mots seuls restent essentiels pour le début de l'apprentissage.
\item L'implémentation d'un clavier spécifique à l'application est nécessaire. Il doit être sous forme \textit{alphabet} et non \textit{azerty}, et contenir les accents et caractères fréquemment rencontrés tels que \textit{-}, \textit{'}, etc.
\item Le fait que certains des enfants aient des problèmes dans la discrimination des lettres me conforte dans l'utilisation de la police spécifique \textit{OpenDyslexic}.
\item Le son, dans le cas de l'exercice avec les anagrammes, est une composante nécessaire pour la réussite de celui-ci.\\
\end{itemize}

De plus, j'ai posé quelques questions aux enfants concernant l'application dans sa globalité. Je leur ai demandé si celle-ci leur plaisait ou non. Ils ont tous répondu par l'affirmative, une des filles précisant qu'elle trouvait plus chouette d'apprendre sur une tablette.\\
Enfin, je leur ai demandé s'ils aimeraient avoir de la musique en plus, comme dans un jeu, ou s'il pensaient que ça les distrairait. Tous m'ont dit qu'il préféraient ne pas avoir de musique, car il leur est plus facile de se concentrer dans le silence.