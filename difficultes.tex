\section{Difficultés rencontrées\label{diff}}
Créer une application Android de A à Z pour la première fois ne se déroule bien évidemment pas sans rencontrer quelques difficultés. Que ce soit pour la programmation ou pour le reste du travail, rien ne se réalise jamais sans rencontrer quelques obstacles. Ce point du rapport est dédié aux petits problèmes et autres difficultés auxquels j'ai été confrontés tout au long de mon travail.\\

La difficulté principale, lorsqu'un projet de l'ampleur de ce travail de fin d'études est réalisé, c'est de trouver la motivation. En effet, bien que le sujet choisi m'intéresse, ce n'était pas facile tous les jours. Durant les premiers mois de sa réalisation, avec le stage en parallèle, je rentrais souvent fatiguée. Dans ces cas là, il faut trouver l'énergie de se mettre à travailler en soirée et réussir à se motiver. Les autres jours, weekends ou après le stage, il m'arrivait d'avoir des jours "sans", comme tout le monde. Enfin, parfois ma démotivation était due à un obstacle rencontré dans mon travail, comme je vais en décrire plusieurs par la suite. Néanmoins, lorsqu'une baisse de motivation se faisait sentir, je pensais au diplôme à la clé, et je me disais que cela vaut donc vraiment la peine de travailler.\\

Le premier problème technique auquel j'ai dû faire face a été le bug d'Eclipse. Comme je l'ai déjà expliqué précédemment, ce bug est apparu lors de l'installation du plugin Maven pour Eclipse. Alors que celle-ci s'était déroulée correctement, j'ai dû redémarrer Eclipse. À partir de là, je n'ai plus su travailler avec cet IDE, car il refusait de fonctionner après seulement quelques minutes. J'ai essayé de résoudre ce problème, sans succès. Je me suis donc résolue à transférer mon projet sous Android Studio et à abandonner Eclipse.\\

Une autre difficulté rencontrée concerne l'enregistrement des sons utilisés dans l'application. Ce n'est pas l'enregistrement en lui même qui m'a posé problème, bien qu'il m'ait pris pas mal de temps. La difficulté était dans la prononciation des mots. Lors de l'enregistrement du VOB, il fallait que je fasse attention à prononcer les mots le mieux possible, sans pour autant tomber dans l'excès. J'ai du m'y reprendre plusieurs fois pour certains d'entre eux.\\

Ensuite, j'ai rencontré quelques soucis lorsque j'ai implémenté le clavier \textit{alphabet} pour l'exercice de lecture flash. Comme expliqué au point \ref{clavier}, j'ai utilisé le code source disponible avec le tutoriel de Martin Pennings. Cependant, avant de remarquer que le code était disponible et ainsi me simplifier la vie, j'ai essayé d'implémenter le clavier sur base du tutoriel uniquement. Pendant quelques heures, je me suis appliquée, pour au final me retrouver avec du code qui fonctionnait aléatoirement. Effectivement, une fois que j'ai vu le code original, les points que j'avais eu du mal à comprendre se sont éclaircis. Dès lors, j'ai opté pour la solution de facilité : supprimer mon code défectueux, ajouter le code original à l'application, et modifier ce dernier pour répondre à mes besoins.\\

L'exercice des anagrammes m'a aussi posé des problèmes. Celui qui m'a accaparée le plus longtemps est la mise en place des boutons de validation et des lettres mélangées. Il faut savoir que lorsque le \textit{listener} permettant le mouvement des lettres est implémenté, celui-ci ne fonctionne que dans l'élément layout dans lequel il est défini\footnote{Un layout (= une vue) peut être composé de plusieurs éléments layout s'imbriquant les uns dans les autres}. Par exemple, il est possible de mettre en place un \textit{LinearLayout} (type structure de layout, un peu comme une grosse boîte) global vertical avec encore deux \textit{LinearLayout} horizontaux (petites boîtes à l'intérieur de la grosse), tout ça pour obtenir un seul et même layout (vue) à afficher sur l'écran. Si les lettres se trouvent dans un des \textit{LinearLayout} interne, elles ne pourront pas se déplacer au dessus du second. Pour éviter cette contrainte, j'ai donc fait le choix d'utiliser une seule "grosse boîte". Ceci rendait impossible l'utilisation du \textit{LinearLayout}, compte tenu de la disposition des éléments de l'interface. J'ai tout d'abord opté pour un \textit{GridLayout}, qui permet d'organiser les éléments sous forme de grille. Cependant, celui-ci ne me permettait pas de centrer le contenu de l'interface comme je le désirais, je suis donc passée au \textit{RelativeLayout}. Celui-ci permet de décrire la position des éléments de manière relative par rapport au parent et aux autres éléments de la vue. Ce type d'élément layout m'a pris un peu de temps à prendre en main, mais répond finalement à mes attentes.\\

L'autre difficulté des anagrammes a été la mise en place de l'algorithme de validation de la position des lettres. En premier lieu, il m'a fallu comprendre comment le \textit{listener} permettant de bouger les lettres fonctionnait. A partir de cela, j'ai rusé pour réussir à valider les positions des boutons, le fait qu'ils soient au dessus de la bonne \textit{case de validation}, etc. Malgré mes idées d'algorithme réfléchies auparavant, et inscrites sur papier à côté de moi, j'ai dû faire plusieurs essais avant d'y parvenir. \\

Enfin, je voudrais mentionner que respecter le planning n'a pas été aussi aisé que je l'aurais imaginé. Je n'ai jamais été une grande fanatique des plannings. Je trouve qu'il est d'autant plus dur de se donner des objectifs à atteindre et de les quantifier dans le temps que l'échéance est éloignée. Ajoutons à cela des baisses de motivation, des imprévus, des soucis, etc, et le planning en a quelque peu souffert. Malgré cela, j'ai rattrapé autant que possible le retard pris, et j'ai fait de mon mieux afin d'atteindre les objectifs que je me suis fixés.


%- problème de planning