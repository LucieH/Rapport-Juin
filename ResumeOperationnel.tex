\documentclass[11pt]{article}
\usepackage[frenchb]{babel}
\usepackage{fontenc}
\usepackage{fancyhdr} % Required for custom headers
\usepackage{lastpage} % Required to determine the last page for the footer
\usepackage{extramarks} % Required for headers and footers
\usepackage{graphicx} % Required to insert images
\usepackage[utf8]{inputenc}
\usepackage{url}
\usepackage[normalem]{ulem}
\usepackage{geometry}
\usepackage{listings}
\usepackage{numprint}

\geometry{a4paper,
		  body={160mm,250mm},
		  left=20mm,
		  top=25mm,
		  bottom=25mm,
		  right=15mm,
		  headheight=10mm,
		  headsep=4mm,
		  marginparsep=4mm,
		  marginparwidth=21mm}
\pagestyle{fancy}
\lhead{\AuthorName}
\rhead{\Date} % Top left header
\chead{\Logo}
\lfoot{}
\rfoot{}
\cfoot{}

%\vspace{}
\newcommand{\AuthorName}{Lucie Herrier - 3TL1} % Your name
\newcommand{\Date}{Rapporteur : V. Van den Schrieck}
\newcommand{\Logo}{\includegraphics[width=2cm]{ephec.png}}


\title{
\vspace{-2cm}
\parbox{15cm}
{	\vspace{0.5cm}
	\begin{center}\sf\bfseries
		Une application Android au service de l'apprentissage de la lecture chez l'enfant : résumé opérationnel
	\end{center}
	\vspace{-2cm}
}} 

\begin{document}
\date{}
\maketitle
\thispagestyle{fancy}
\section{Objet du travail de fin d'études}
Ce travail de fin d'études est consacré à la conception d'une application Android afin d'aider les enfants à apprendre à lire. Cette application est destinée aux enfants qui débutent dans l'apprentissage de lecture et dont le niveau n'est pas trop élevé, vers la première primaire ou avant. Elle sert de complément et d'entraînement sur le côté de l'école, par exemple à la maison. Par ailleurs, celle-ci répond à un manque d'applications complètes et variées en terme d'exercices, et basées sur les conseils de professionnels, sur le marché Android.

\section{Méthode de travail}
Avant de démarrer le développement proprement dit de l'application, une analyse de la problématique a été effectuée. J'ai rencontré une institutrice primaire et une logopède afin de me renseigner sur les méthodes d'apprentissage de la lecture chez l'enfant. Sur la base des informations obtenues, j'ai effectué une étude comparative avec les produits existants sur le marché. Avec les données de cette analyse, j'ai mis en place un cahier des charges définissant des objectifs à atteindre. J'ai ensuite développé l'application à partir de ces derniers, exercice par exercice.\\

J'ai utilisé différents outils lors de mon travail :
\begin{itemize}
\item comme IDE : Android Studio,
\item Adobe Illustrator pour les graphismes,
\item Audacity pour l'enregitrement et l'édition audio, 
\item OpenDyslexic, une police de caractère open source,
\item le VOB (Vocabulaire Orthographique de Base) du cycle inférieur pour le vocabulaire des exercices.
\item un smartphone et des tablettes sous Android, depuis la version 4.0 jusqu'à 5.1, pour les tests.
\end{itemize}

\section{Résultats obtenus}
L'application développée s'appelle \textit{Manabu}, ce qui signifie apprendre en japonais. Elle respecte le cahier des charges et est basée sur les méthodes d'apprentissage utilisées à l'école et en logopédie : la méthode syllabique (apprendre à lire sur la base des sons des lettres et des syllabes) et la méthode naturelle (apprendre à lire à l'aide de sujets intéressants et qui font appel à la créativité de l'enfant, par exemple avec des images, couleurs, etc.). \textit{Manabu} propose quatre exercices différents aux enfants : \textit{Imagerie} (association mot-image), \textit{Lecture flash} (lecture rapide et restitution), \textit{Anagrammes} (remise de lettres dans l'ordre) et \textit{Écouter le son} (association d'un son de syllabe ou lettre avec un mot). De plus, chaque exercice est disponible en trois niveaux de difficulté : facile, moyen, et difficile. Enfin, l'application est conçue pour être à la fois conviviale, amusante, et éducative.

\section{Esquisse des recommandations}
L'application \textit{Manabu} est prévue pour fonctionner sur tablette à partir de l'API 14 d'Android (4.0 Ice Cream Sandwich). Si la version d'Android est inférieure, le fonctionnement n'est pas garanti. Elle tourne également sur smartphone, mais les écrans de ceux-ci sont plus petits et moins confortables. \textit{Manabu} est destinée aux enfants débutants dans l'apprentissage de la lecture ou ayant des difficultés avec celle-ci.



\end{document}






	