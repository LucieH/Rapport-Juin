\section{Introduction : énoncé de la thématique du TFE et motivations}
Le rôle des nouvelles technologies dans l'apprentissage chez les enfants est un sujet qui m'inté-resse tout particulièrement. Je me suis souvent interrogée sur les bienfaits, ou méfaits selon certains, de l'utilisation d'appareils informatiques chez les plus jeunes. De ce fait, j'ai décidé de consacrer mon travail de fin d'études au développement d'une application Android afin d'aider et de compléter l'apprentissage de la lecture chez les enfants, en parallèle de l'école. La programmation Android est d'autant plus actuelle que le nombre de tablettes et de smartphones est en constante augmentation, et que la majorité de ces appareils tourne sous le célèbre système d'exploitation de Google. La pertinence de cette problématique a d'ailleurs été confirmée lors des travaux préparatoires de ce travail de fin d'études : parmi les applications disponibles, peu ont été réalisées en se basant sur les précieux conseils de professionnels du secteur de l'apprentissage.\\

Ce travail de développement d'application s'appuie essentiellement sur l'analyse des informations obtenues auprès d'une institutrice et d'une logopède. Il a également été tenu compte de l'avis des enfants. Intitulé \textit{Une application Android au service de l'apprentissage de la lecture chez l'enfant}, ce travail de fin d'étude s'entend à démontrer qu'Android peut aider les enfants à apprendre, conjointement à la méthode traditionnelle employée par les instituteurs. En effet, une application permet aux enfants de s'entraîner à l'apprentissage de la lecture, tout en s'amusant.\\

La première partie est consacrée à l'analyse du problème, où j'expose ce que j'ai appris lors de mes entretiens avec les personnes travaillant dans le domaine de l'apprentissage de la lecture chez l'enfant, ainsi qu'une étude comparative du marché et un cahier des charges. Le point suivant consiste en une présentation de l'application ainsi que des exercices qui s'y retrouvent. Dans cette partie, j'explique les choix auxquels j'ai procédé afin de mettre en place des exercices reflétant au mieux les conseils obtenus précédemment. Par la suite, je présente les outils et technologies que j'ai utilisés afin de mener à bien ce projet, avant d'expliciter la manière dont s'est déroulé le développement Android à proprement parler. J'expose ensuite les résultats du test de l'application par des enfants. Enfin, je termine ce rapport par une analyse des difficultés que j'ai pu rencontrer tout au long de ce travail de fin d'études, ainsi que des améliorations qu'il serait possible d'apporter à l'application Android, avant de conclure.