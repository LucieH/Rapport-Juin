\section{Introduction}
Le rôle des nouvelles technologies dans l'apprentissage chez les enfants est un sujet qui m'intéresse tout particulièrement. Je me suis souvent interrogée sur les bienfaits, ou méfaits selon certains, de l'utilisation d'appareils informatiques chez les plus jeune. De ce fait, j'ai décidé de consacrer mon travail de fin d'études au développement d'une application android afin d'aider et complémenter l'apprentissage de la lecture chez les enfants, en parallèle à l'école. La programmation android est d'autant plus actuelle que le nombre de tablettes et de smartphones est en constante augmentation, et que la majorité de ces appareils tournent sous le célèbre système d'exploitation de Google. La pertinence de cette problématique s'est d'ailleurs confirmée au cours des travaux préparatoires de ce travail de fin d'études : parmi les applications disponibles, peu d'entre elles ont été réalisées en se basant sur les précieux conseils de professionnels du secteur de l'apprentissage.\\

Ce travail de développement d'application s'appuie essentiellement sur l'analyse des informations obtenues auprès d'une institutrice, d'une logopède, et même de l'avis des enfants. Intitulé "Une application android au service de l'apprentissage de la lecture chez l'enfant ", ce travail de fin d'étude tend à démontrer qu'il est possible qu'android aide les enfants à apprendre, conjointement à la méthode traditionnelle employée par les instituteurs. En effet, un application peut entraîner les enfants dans l'apprentissage de la lecture, tout en s'amusant.\\

Après une première partie consacrée à la recherche documentaire, où l'on observe que j'y expose ce que j'ai appris lors de mes entretiens avec les personnes travaillant dans le domaine de l'apprentissage de la lecture chez l'enfant, le point suivant est consacré à une présentation de l'application ainsi que des exercices qui s'y retrouven. Celui-ci explique les choix auxquels j'ai procédé afin de mettre en place des exercices reflétant au mieux les conseils obtenus précédemment. Par la suite, je présente les outils et technologies que j'ai utilisés afin de mener à bien ce projet, avant d'expliciter la manière dont s'est déroulé le développement android à proprement parler. Enfin, je termine ce rapport par une analyse des difficultés que j'ai pu rencontrer tout au long de ce travail de fin d'études, ainsi que les améliorations qu'il serait possible d'apporter à l'application android, avant de conclure.