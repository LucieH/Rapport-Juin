\section{Pistes d'amélioration}
Avant d'arriver à la fin de ce rapport, j'aimerais parler des améliorations qu'il  serait possible d'ajouter à \textit{Manabu}. Ce rapport étant rendu deux semaines avant la défense, je commencerai par expliquer ce que j'espère apporter comme complément d'ici la date fatidique du 19 juin. J'explorerai ensuite diverses pistes d'amélioration qui pourraient rendre mon application encore plus attrayante et efficace.\\

J'aimerais présenter une application complète telle que celle que j'avais imaginée au départ. Quatre exercices, et pour chacun d'entre eux, trois niveaux de difficulté. A l'heure où j'écris ces lignes, les quatre exercices sont tous fonctionnels pour le niveau 1 : facile. J'aimerais donc implémenter les niveaux moyen et difficile pour tous. C'est déjà le cas pour \textit{Anagrammes}, et je suis sûre de pouvoir y arriver au moins pour deux des autres exercices : \textit{Lecture Flash} et \textit{Écouter le son}.\\

Par ailleurs, je ne suis pas entièrement satisfaite de l'aspect graphique de l'application. Je compte retravailler certains layouts afin qu'ils soient plus agréables visuellement. J'aimerais, si j'ai le temps, ajouter la mascotte dont j'avais parlé dans le cahier des charges. Je trouve que celle-ci apporterait un intérêt ludique supplémentaire. Personnellement, je pense que les enfants peuvent s'attacher à une mascotte.\\

Concernant les pistes d'amélioration supplémentaires, je pense que rien n'est jamais parfait. Je pourrais trouver énormément de choses à rajouter, de détails à améliorer, mais ce n'est pas le but ici, et ce ne sera pas le cas. Je vais simplement proposer quelques idées qui apporteraient véritablement quelque chose à \textit{Manabu}.\\

La premier ajout serait d'enregistrer les scores des enfants pour les différents exercices. L'enregistrement des scores est selon moi source de controverses. Certains pensent que cela permet à l'enfant de mesurer sa progression et qu'ils s'agit de données intéressantes. Pour ma part, j'ai choisi de ne pas implémenter de scores car j'estime qu'il s'agit de quelque chose qui peut stresser l'enfant. En effet, si deux enfants comparent leurs scores respectifs, si l'un est nettement moins avancé, il peut se sentir défavorisé. Or, chaque enfant avance à son rythme. Je conçois qu'enregistrer les scores puisse être intéressant, cependant, je ne vois pas l'utilité de mettre une pression supplémentaire sur les épaules de l'enfant. Je proposerais plutôt un système d'\textit{achievements}, par exemple :
\begin{itemize}
\item la série d'anagrammes résolue en moins de X minutes,
\item achèvement du niveau facile, ce pour chaque exercice et chaque niveau,
\item niveau facile de la lecture flash réussi avec le temps de lecture minimal,
\item etc.
\end{itemize}

Une deuxième amélioration possible serait la personnalisation des différents exercices en fonction des centres d'intérêt de l'enfant. Par exemple, si un enfant adore le sujet de l'espace et que l'exercice \textit{Imagerie} est son favori, on peut imaginer acheter une extension pour cette exercice. Celle-ci contiendrait des images et de nouveaux mots en rapport avec le thème de l'espace à apprendre à l'enfant. Ce dernier serait alors plus prompt à utiliser l'application, celle-ci étant en rapport avec ses centres d'intérêt. On peut bien entendu imaginer cette personnalisation pour chacun des exercices présents dans \textit{Manabu}.\\


Enfin, la troisième amélioration possible découle d'une idée énoncée de la rencontre avec Laurence Henrion et concerne l'exercice \textit{Lecture flash}. Il s'agit simplement de pouvoir ajouter manuellement un mot parmi ceux déjà existants dans l'exercice. Ceci permettrait à un adulte responsable (parent ou autre) d'ajouter du nouveau vocabulaire. Ce dernier serait bien évidemment mémorisé pour les futures utilisations de \textit{Manabu}. Il s'agit d'une alternative au téléchargement d'extensions mentionné dans le paragraphe précédent, et qui augmente les possibilités de personnalisation.
