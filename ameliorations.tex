\section{Pistes d'amélioration}
Avant d'arriver à la fin de ce rapport, j'aimerais parler des améliorations qu'il  serait possible d'ajouter à \textit{Manabu}. Ce rapport étant rendu deux semaines avant la défense, je commencerai par expliquer ce que j'espère apporter comme complément d'ici la date fatidique du 19 juin. J'explorerai ensuite diverses pistes d'amélioration qui pourraient rendre mon application encore plus attrayante et efficace.\\

J'aimerais présenter une application complète telle que celle que j'avais imaginée au départ. Quatre exercices, et pour chacun d'entre eux, trois niveaux de difficulté. A l'heure où j'écris ces lignes, les quatre exercices sont tous fonctionnels pour le niveau 1 : facile. J'aimerais donc implémenter les niveaux moyen et difficile pour tous. Je suis sûre de pouvoir y arriver au moins pour deux des exercices : \textit{Lecture Flash} et \textit{Anagrammes}.\\

Par ailleurs, je ne suis pas entièrement satisfaite de l'aspect graphique de l'application. Je compte retravailler certains layouts afin qu'ils soient plus agréables visuellement. J'aimerais, si j'ai le temps, ajouter la mascotte dont j'avais parlé dans le cahier des charges. Je trouve que celle-ci apporterait un intérêt ludique supplémentaire. Personnellement, je pense que les enfants peuvent s'attacher à une mascotte.\\

Concernant les pistes d'amélioration supplémentaires, je pense que rien n'est jamais parfait. Je pourrais trouver énormément de choses à rajouter, de détails à améliorer, mais ce n'est pas le but ici, et ce ne sera pas le cas. Je vais simplement proposer quelques idées qui apporteraient véritablement quelque chose à \textit{Manabu}.\\

-enregistrement des scores\\

-ajout de contenu spécialisé en fonction des centres d'intérêt de l'enfant (imagerie, lecture flash, anagrammes, sons) => customisation de l'appli\\

-pouvoir ajouter des mots à la main dans la lecture flash.